\section{Persistenzschicht mit Hibernate} \label{sec:impl-Persistenzschichten}

Die Implementation der Persistenzschicht unter Verwendung von Hibernate wurde in mehreren Schritten vollzogen:

\begin{enumerate}

\item Erzeugen der Hibernate-Mapping-Dateien
\item Generieren der Mapping-Klassen aus den XML-Dateien
\item Generieren des Datenbankschemas
\item Erzeugen der Hibernate-Konfiguration
\item Entwicklung der Klasse \textit{Database} f�r die Interaktion mit der Datenbank

\end{enumerate}

Diese Schritte werden im Folgenden etwas genauer beschrieben.

\subsection{Erzeugen der Hibernate-Mapping-Dateien}

Wie in Kapitel \ref{sec:tech-hibernate} beschrieben, sind die Mapping-Dateien f�r das Abbilden eines Datenbankobjekts auf eine Java-Klasse notwendig.\\
Zun�chst wurde in der Phase der Implementierung ein entsprechendes Datenmodell mit allen ben�tigten Entit�ten und Beziehungen erzeugt. Daraus wurden die entsprechenden Hibernate-Mapping-Dateien gebildet, indem hier unter Angabe von Name, Typ, Name der korrespondierenden Tabellenspalte der Datenbank und einigen anderen zum Teil auch optionalen Attributen alle Properties eines persistenten Objekts definiert wurden.
Wichtig ist auch, dass neben den Properties mit einfachen Datentypen auch s�mtliche Beziehungen zu anderen Objekten bekannt gemacht werden mu�ten.

\subsection{Generieren der Mapping-Klassen aus den XML-Dateien}

Nachdem alle Mapping-Dateien erzeugt wurden, war es notwendig daraus die entsprechenden Java-Klassen erzeugen zu lassen. Hierf�r wurde das Tool zur Code-Generierung von Hibernate verwendet. Die Automatisierung dieses Prozesses wurde in ein Ant Build-Skript eingef�gt, indem das Tool mit den entsprechenden Optionen von diesem Skript aus ausgef�hrt werden kann.\\
Wie in \vref{code:generate_classes} ersichtlich, wird �ber das Element \textit{taskdef} das Tool in das Skript integriert und �ber \textit{target} die Anweisungen zur Ausf�hrung definiert.
Mithilfe des Ant-Plugins f�r Eclipse konnte dieses Skript zur Ausf�hrung gebracht werden.

\begin{lstlisting}[language=XML, caption={Erzeugung der persistenten Java-Klassen �ber Apache Ant}, label=code:generate_classes, showstringspaces=false]
    <!-- Teach Ant how to use Hibernate's code generation tool -->
    <taskdef name="hbm2java"
             classname="net.sf.hibernate.tool.hbm2java.Hbm2JavaTask"
             classpathref="project.class.path"/>

    <!-- Generate the java code for all mapping files in our source tree -->
    <target name="codegen"
             description="Generate Java source from the O/R mapping files">
        <hbm2java output="${source.root}">
          <fileset dir="${source.root}">
            <include name="**/*.hbm.xml"/>
          </fileset>
        </hbm2java>
    </target>
\end{lstlisting}

\subsection{Generieren des Datenbankschemas}

F�r die Verwaltung der Provirent-Datenbank wurde die Verwendung des DBMS Firebird vorgezogen.
Dabei wurde ein entsprechender Server auf den Entwicklungsrechnern installiert. Mithilfe des Firebird-Clients FlameRobin war der projekt-externe Zugriff auf diesen Datenbankserver und dementsprechend auf die Datenbank m�glich.\\
Zuvor musste jedoch erst das Datenbankschema generiert werden. Hierbei wurde das Hibernate-Tool zur Generierung von Datenbankschemata verwendet. Wie bei der Generierung der persistenten Java-Klassen wurden die Anweisungen f�r die Generierung in ein Ant Build-Skript, wie in \vref{code:generate_schema} zu sehen, integriert.

\begin{lstlisting}[language=XML, caption={Erzeugung des Datenbankschemas �ber Apache Ant}, label=code:generate_schema, showstringspaces=false]
    <!-- Generate the schemas for all mapping files in our class tree -->
    <target name="schema" depends="compile"
            description="Generate DB schema from the O/R mapping files">

      <!-- Teach Ant how to use Hibernate's schema generation tool -->
      <taskdef name="schemaexport"
               classname="net.sf.hibernate.tool.hbm2ddl.SchemaExportTask"
               classpathref="project.class.path"/>

      <schemaexport properties="\${class.root}/hibernate.properties" quiet="no" text="no" drop="no" delimiter=";">
        <fileset dir="\${class.root}">
          <include name="**/*.hbm.xml"/>
        </fileset>
      </schemaexport>
    </target>
\end{lstlisting}

\subsection{Erzeugen der Hibernate-Konfiguration}

Die Konfiguration des Hibernate-Frameworks bedurfte es, zum einen die Eigenschaften der Datenbankverbindung zu deklarieren und zum anderen die verschiedenen Mapping-Dateien dem Framework bekannt zu machen.\\
Die Datenbankverbindung wurde in einer Properties-Datei (siehe \vref{code:hibernate_properties}) beschrieben. Dabei war es unter anderem notwendig, folgende Eigenschaften zu definieren:

\begin{itemize}

\item Dialekt der Datenbank
\item Datenbanktreiber
\item URL der Datenbank
\item Benutzername und Passwort f�r die DB

\end{itemize}

\begin{lstlisting}[language=XML, caption={hibernate.properties}, label=code:hibernate_properties, showstringspaces=false]

hibernate.dialect=net.sf.hibernate.dialect.FirebirdDialect
hibernate.connection.driver_class=org.firebirdsql.jdbc.FBDriver
hibernate.connection.url=jdbc:firebirdsql:localhost:c:/video
hibernate.connection.username=SYSDBA
hibernate.connection.password=masterkey
hibernate.jdbc.use_streams_for_binary=true

\end{lstlisting}

Zus�tzlich dazu musste noch eine Klasse implementiert werden, die Sessions f�r das Hibernate-Framework bereitstellt. Mithilfe der Klasse \textit{HibernateUtil} wurden nicht nur Methoden zum Erzeugen und Beenden von Sessions angeboten, sondern auch das objektrelationale Mapping der Provirent-Datenbank f�r Hibernate sichtbar gemacht. Hierbei wird eine SessionFactory erzeugt, dessen Konfiguration die Namen aller persistenten Klassen besitzt und somit Zugriff darauf hat.

\subsection{Entwicklung der Klasse \textit{Database} f�r die Interaktion mit der Datenbank}

Nachdem nun das Hibernate-Framework alle entsprechenden Einstellungen erhalten hat, war es m�glich dar�ber mit der Datenbank zu interagieren. Um jedoch eine zentrale Klasse mit allen Operationen zu erhalten, wurde die Klasse \textit{Database} implementiert. Hier befinden sich z.B. Methoden zum Laden, Speichern, Aktualisieren oder L�schen von einzelnen oder mehreren Datenbankobjekten.\\
Die Methode \textit{getSingleActor}() in \vref{code:method_getSingleActor} l�dt einen Schauspieler anhand seiner ID aus der Datenbank und gibt diesen als Objekt der Klasse \textit{Actor} zur�ck.

\begin{lstlisting}[language=Java, caption={Methode zum Laden eines Schauspielers aus der DB}, label=code:method_getSingleActor, showstringspaces=false]
public static Actor getSingleActor(final int id) {
	if (logger.isDebugEnabled()) {
		logger.debug("getSingleActor() - start. int filter= " + id);
	}
	//init the returnlist
	Actor returnobject = null;

	Session s = null;
	Transaction tx = null;
	try {
		//get new Session and begin Transaction
		s = HibernateUtil.currentSession();

		returnobject = (Actor) s.get(Actor.class, new Integer(id));

	} catch (Exception e) {
		logger.error("getSingleActor() - Error while trying to do Transaction", e);

	} finally {
		try {
			// No matter what, close the session
			HibernateUtil.closeSession();
		} catch (HibernateException e1) {
			logger.error("getSingleActor() - Could not Close the Session", e1);
		}
	}

	if (logger.isDebugEnabled()) {
		logger.debug("getSingleActor() - end");
	}
	return returnobject;
}
\end{lstlisting}














%Hier danach nicht mehr schreiben
\label{sec:impl-Persistenzschichten-ende}