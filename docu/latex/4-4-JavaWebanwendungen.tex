\section{Java-Web-Anwendung mit Struts} \label{sec:impl-WebAnwendungen}

Die Webanwendung ist daf�r gedacht, dass Kunden von �berall die M�glichkeit (mit der Voraussetzung eines Internetzugangs) haben sollen, nach DVDs im Datenbestand zu suchen, Informationen dar�ber abzurufen und diese zu bestellen.\\
Der aktuelle Stand dieser Webanwendung ist insofern implementiert, dass sich bereits Kunden in das System einloggen k�nnen. Anhand dieser Funktionalit�t wird auf die Implementierung unter Verwendung des Web-Frameworks Apache Struts eingegangen.\\
Folgende Schritte waren dabei notwendig:

\begin{itemize}

\item Login-Formular in JSP integrieren
\item ActionForm Bean f�r das Formular erzeugen
\item Action-Klasse f�r das Login erzeugen
\item ActionForm und Action in struts-config.xml registrieren

\end{itemize}

\subsection{Login-Formular}

Um den Kunden die M�glichkeit zu geben, sich am System anzumelden, gibt es ein entsprechendes Formular. Darin muss der Benutzer seinen Benutzernamen sowie sein pers�nliches Passwort eingeben mit Best�tigung seiner Eingaben �ber den Login-Button soll diese Anfrage mit den Daten in der Datenbank verglichen werden und bei Erfolg eine dem Benutzer angepassten Seite geladen werden.\\
Das Login-Formular wird in \vref{code:login_jsp} mit den entsprechenden Tags der Struts-Taglibrary gebildet. Das \verb|<html:form>|-Tag grenzt darin das Formular ein. �ber das Attribut \textit{action} wird dem Servlet signalisiert, dass die in der \textit{struts-config.xml} definierte Action mit dem Pfad \textit{/Login} beim Klick des Submit-Buttons ausgef�hrt werden soll.\\
Mit der Verwendung von \verb|<html:text>|-Tags werden dem Benutzer Eingabefelder erzeugt, worin dieser seine Daten zum Einloggen eintragen kann. �ber das Attribut \textit{property} wird auf das entsprechende Attribut in der ActionForm Bean referenziert. Das bedeutet bei einer Bet�tigung des Login-Buttons, der durch das Tag \verb|<html:submit| dargestellt wird, dass die Instanz der ActionForm Bean mit den eingegebenen Daten aus dem Formular versehen wird.

\begin{lstlisting}[language=XML, caption={Login-Formular in der JSP}, label=code:login_jsp, showstringspaces=false]
<html:form action="/Login" method="post">
  <tr>
    <td>
      <label class="login_label"><bean:message key="main.login.username"/>:</label>
      <html:text property="username" styleClass="login" style="font-weight: bold;" size="13"/>
    </td>
    <td>
      <label class="login_label"><bean:message key="main.login.passwort"/>:</label>
      <html:password property="password" styleClass="login" size="14" />
    </td>
    <td>
       <html:submit title="main.login.submit" styleId="submit_search"/>
    </td>
  </tr>
</html:form>
\end{lstlisting}

\subsection{Erzeugung der ActionForm Bean}

Die Klasse zur Aufbewahrung der Daten eines Formulars besteht prinzipiell aus den Attributen und ihren Getter- bzw. Setter-Methoden. Dementsprechend wurde die Klasse \textit{LoginForm} mit den Attributen \textit{username} und \textit{password} implementiert. Die Methoden zum Erhalt bzw. Setzen der Attribute wurden in diesem Fall noch durch zwei weitere Methoden zum Validieren der Eingaben bzw. zum Zur�cksetzen der Formulareingaben komplettiert.


\subsection{Erzeugung der Action-Klasse}

Zur Ausf�hrung der Gesch�ftslogik, in diesem Anwendungsfall des Logins, war es notwendig die Klasse \textit{LoginAction}, die von der Klasse \textit{Action} aus dem Struts-Framework erbt, zu implementieren. Die Methode \textit{execute}() spielt dabei die gr��te Rolle, da diese zwingend implementiert werden musste. Darin wird die Logik ausgef�hrt, um den Kunden zu identifizieren und authentifizieren. In dieser Methode (siehe \vref{code:login_action}) werden zun�chst die Daten aus der ActionForm Bean ausgelesen, d.h. Benutzername und Passwort. Unter Verwendung der Klasse \textit{Database}, die im Kapitel \ref{sec:impl-Persistenzschichten} beschrieben wird, werden die Benutzerdaten mit denen aus der Datenbank verglichen. Bei erfolgreichem Ergebnis wird das aus der Datenbank geladene \textit{Customer}-Objekt (gleichzusetzen mit dem Kunden des Systems) in die Session der Webanwendung gespeichert und eine entsprechende Antwort (Response) an den Browser gesendet.

\begin{lstlisting}[language=Java, caption={execute()-Methode der LoginAction}, label=code:login_action, showstringspaces=false]
public ActionForward execute(
			ActionMapping mapping, ActionForm form,
			HttpServletRequest request,
			HttpServletResponse response)
      	throws Exception {
    
  LoginForm lf = (LoginForm) form;
		
  boolean existUser = false;
		
  //	Laden von Benutzername und Passwort
  String aktUsername = lf.getUsername();
  String aktPassword = lf.getPassword();
		
  Customer user = null;
		
  List users = Database.getUsersByLogin(aktUsername, aktPassword);
		
  //�ber die Benutzerliste iterieren
  while (users.iterator().hasNext()) {

    existUser = true;
			
    //Benutzer aus Liste laden
    user = (Customer) it.next();
  }

  //		wenn Benutzer nicht existiert
  if (!existUser) {
  
    //Eingabeformular laden
    return mapping.getInputForward();
    
  } else {
  
    request.getSession().setAttribute(StrutsConstant.SESSION_USER_KEY, user);	
    return mapping.findForward(StrutsConstant.FWD_SUCCESS);	
    		
  }
}
\end{lstlisting}

\subsection{Eintr�ge in die \textit{struts-config.xml}}

Die Zuordnung von AcctionForm Bean und Action-Klasse erfolgt in der Struts-Konfigurationsdatei. Zum einen musste die Form Bean registriert werden:

\begin{lstlisting}[language=XML, showstringspaces=false]
<form-bean name="LoginForm" type="de.hsharz.provirent.customer.form.LoginForm"/>
\end{lstlisting}

Ferner wurde hier ein entsprechendes Action-Mapping definiert:

\begin{lstlisting}[language=XML, showstringspaces=false]
<action path="/Login" 
	input="provirent.index" 
	type="de.hsharz.provirent.customer.action.LoginAction"
	name="LoginForm"
	validate="true" >
  <forward name="success" path="/index.do"/>
</action>
\end{lstlisting}

In diesem Mapping erfolgt zum einen die Zuordnung der Form Bean \textit{LoginForm} zur Action \textit{LoginAction} und zum anderen wird der Forward, also das Weiterleiten des Requests, bei erfolgreichem Login definiert.



%Hier danach nicht mehr schreiben
\label{sec:impl-WebAnwendungen-ende}