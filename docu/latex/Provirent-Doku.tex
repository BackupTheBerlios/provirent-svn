\documentclass[12pt,a4paper]{report}
%\documentclass[17pt]{extreport}
\usepackage[latin1]{inputenc}
\usepackage{ngerman}
\usepackage{makeidx,longtable,geometry,fancyhdr,color}
\usepackage{colortbl}
\usepackage[]{graphicx}
%-- dadurch werden zweizeilige Fussnoten einger�ckt
\usepackage[hang]{footmisc}
%-- refpage und bei dem Abk�rzungsverzeichniss steht die Seite auf der die Abk�rzung definiert wird
\usepackage[german,refpage]{nomencl}



\usepackage{enumitem}
\setlist{noitemsep,topsep=0.25cm}

\definecolor{dunkelgrau}{gray}{0.55}
\definecolor{hellgrau}{gray}{0.90}

\usepackage{listings}
\lstset{basicstyle=\scriptsize\ttfamily, numbers=left, numberstyle=\tiny, stepnumber=1,numbersep=10pt, firstnumber=1, backgroundcolor=\color{hellgrau}, frame=shadowbox, rulesepcolor=\color{dunkelgrau},
framesep=5pt, framerule=0.75pt, captionpos=b, breaklines=true, breakatwhitespace=false, escapeinside={!*}{*!}}

\renewcommand{\lstlistlistingname}{Quellcode Verzeichnis}
\renewcommand{\lstlistingname}{Quellcode}


% variable referenzennamen
\usepackage[german]{varioref}
\labelformat{lstlisting}{Quellcode~#1}
\labelformat{chapter}{Kapitel~#1}
\labelformat{section}{Abschnitt~#1}
\labelformat{subsection}{Unterabschnitt~#1}
\labelformat{subsubsection}{Unterunterabschnitt~#1}
\labelformat{paragraph}{Absatz~#1}
\labelformat{subparagraph}{Unterabsatz~#1}
\labelformat{figure}{Abbildung~#1}
\labelformat{table}{Tabelle~#1}
\labelformat{footnote}{Fussnote~#1}







%----Fuer Quellenangaben
\usepackage[square]{natbib}
\citestyle{natdin}
\bibliographystyle{natdin}





\usepackage{hyperref}
\hypersetup{% �=\304; �=\326; �=\334; �=\344; �=\366; �=\374; �=\377
  pdftitle={Provirent - Dokumentation},
  pdfauthor={Philipp Schneider},
  pdfcreator={Creator Philipp Schneider},
  pdfproducer={Producer Philipp Schneider},
  pdfsubject={},
  pdfkeywords={},
  colorlinks=false,   
  pdfpagemode=true,
  breaklinks=true,
  linkcolor = black, % blue
  anchorcolor=black % green
}


\pagestyle{fancy}
\renewcommand{\chaptermark}[1]{\markboth{#1}{}}
\renewcommand{\sectionmark}[1]{\markright{\thesection\ #1}}
\fancyhf{} % delete current setting for header and footer

\fancyhead[L]{\bfseries\rightmark}
\fancyhead[R]{}
\fancyfoot[C]{\bfseries\thepage}
\fancyfoot[L]{\bfseries\leftmark}
\fancyfoot[R]{\bfseries \footnotesize Provirent-Doku}

\renewcommand{\headrulewidth}{0.5pt}
\renewcommand{\footrulewidth}{0pt} 
\addtolength{\headheight}{2.5pt} % make space for the rule
\fancypagestyle{plain} {%
	\fancyhead{} % get rid of headers on plain pages
	\renewcommand{\headrulewidth}{0pt} % and the line
} 

%-- Index
\makeindex
\renewcommand{\indexname}{Stichwortverzeichnis} 

%--AbkuerzungsVerzeichnis
\renewcommand{\nomname}{Abk�rzungsverzeichnis}
\makenomenclature

%-- Zweilzeilig zum besseren Korrekturlesen 
%\usepackage{setspace}
%\doublespace      % doppelzeilig
%\onehalfspacing  % anderthalbzeilig

%hier werden die trennungen f�r spezielle W�rter definiert
\hyphenation{
Hal-lo
}

\begin{document}
%------ Begin des Dokumentes ------ 
   	\setlongtables
		\definecolor{Gray}{gray}{0.8}
 		\newcolumntype{E}{|>{\columncolor{Gray}[\tabcolsep]}c|}
 		\newcolumntype{A}{|>{p{8cm}\columncolor{Gray}[\tabcolsep]}l}
 		\newcolumntype{B}{|>{\columncolor{Gray}[\tabcolsep]}c|}
	 	\newcolumntype{C}{|>{\columncolor{Gray}[\tabcolsep]}c|}


%====================================
%\begin{titlepage}
\begin{center}
  {\large{\textsc{Projektarbeit}}}
  \\
  \vspace{4.5cm}
  \Huge{\textsc{provirent} \\ ---}
  \\
  \vspace{0.5cm}
  \Large{\textsf{\textbf{Prof}essional \textbf{Vi}deo \textbf{Rent}al Software}}
  \\
  \vspace*{7.5 cm}
\textsf{
    \begin{large}
    \begin{tabular}{ll}
      Student: & Stefan Forstner\\
      Student: & Remo Griesch\\
      Student: & Philipp Schneider\\
      Fachbereich: & Automatisierung und Informatik \\
      Fachrichtung: & Kommunikationsinformatik \\
      Betreuer: & Prof. Dr. Sigurd G�nther\\
      Abgabedatum: & 7. November 2005
    \end{tabular}
  \end{large}
  }
\end{center}
\end{titlepage}



%\thispagestyle{empty}
\begin{titlepage}
\addtolength{\oddsidemargin}{-1.0cm}
\addtolength{\evensidemargin}{-4.0cm}
\addtolength{\textwidth}{+4.0cm}

\newpage

\vspace*{7.5 cm}
%\begin{center}

\LARGE{
\textsc{
\hspace*{0.5cm}Mache die Dinge so einfach wie m�glich\\[0.5cm]
\hspace*{8cm}---\\[0.5cm]
\hspace*{4.8cm}aber nicht einfacher.\\[1.0cm]
}}

%\end{center}
\large
\hspace*{10cm}\texttt{Albert Einstein}
\end{titlepage}




%\renewcommand{\abstractname}{Abstract}
\abstract Hier folgt eine kurze Zusammenfassung des Themas (ca. 5~
Zeilen).



\chapter*{Vorwort}







\tableofcontents

%\addcontentsline{toc}{chapter}{AbbildungsVerzeichnis}
%\listoffigures
%\addcontentsline{toc}{chapter}{TabellenVerzeichnis}
%\listoftables
%\addcontentsline{toc}{chapter}{Quellcodes}
%\lstlistoflistings
%====================================
\chapter{Einf�hrung} \label{sec:Einfuehrung}



\section{Grundlegendes} \label{sec:Grundlegendes}

\subsection{Betreuender Professor}
%===========================================
Hochschule Harz \\
Prof. Dr. Sigurd G�nther \\
Friedrichstr. 57- 59 \\
38855 Wernigerode \\
sguenther@hs-harz.de
%===========================================
\subsection{Studenten}
%===========================================		
\begin{tabular}{rrr}
	Remo Griesch							& Stefan Forstner 				& Philipp Schneider 				\\
	Strasse										&	Strasse der Jugend 22  	& Kastanienring 16				\\
	Ort												&	04880 Dommitsch					& 04316 Leipzig							\\[0.4cm]
	Romeodied@gmx.de					&	fossiossi@web.de				& provirent@phil-schneider.de \\
\end{tabular}		
			

\newpage
\section{Motivation} \label{sec:Motivation}
Im Rahmen des Studiums an der Fachhochschule Harz in Wernigerode mu� jeder Student des Studiengangs Kommunikationsinformatik eine Projektarbeit abgeben. Dies bedeutet, da� der Student eine Aufgabe (meist Programmieraufgabe) alleine oder in einem kleinen Team bew�ltigen muss. Die Professoren der Hochschule bieten dabei viele interessante Projektarbeiten an, sind jedoch auf offen f�r eigene Vorschl�ge der Studenten.\\
Da schon in den Teamprojekten\footnote{Auch das Teamprojekt ist Bestandteil des Studiums. Beim Teamprojekt m��en mehrere Studenten (7-15) gemeinsam eine Programmieraufgabe umsetzen.} \emph{Labmin}\footnote{\url{http://labmin.de.vu}} und \emph{German Team Sony Aibo}\footnote{\url{http://www.der-baer.com/projects.htm}} eine interessante Aufgabe von den Studenten gel�st wurde, sollte das dort erlernte Wissen vertieft und weiter ausgebaut werden.


\section{Ideen zur Projektarbeit} \label{sec:Ideen}
\subsection{Tippspiel}\label{sec:Tippspiel}
Die erste Idee dieser Projektarbeit war die Umsetzung eines Tippspiels in Java, passend zu den damaligen Fussball-Europameisterschaft in Portugal. Diese Idee wurde im JavaMagazin\footnote{\citep{Frotscher2004}\citep{Frotscher2004a}\citep{Frotscher2004b}} in mehreren Ausgaben aufgegriffen und verschiedene Ansatzm�glichkeiten diskutiert. Die Idee unseres Tippspiel war dabei eine Webanwendung mit Datenbankanbindung. Nutzer dieses Systems sollten sich in verschiedenen Tippgemeinschaften, mit je einem Tippgemeinschaftverwalter, zusammen tun und gemeinsam die EM 2004 tippen. Das Tippspiel sollte jedoch nicht nur auf die EM 2004 zugeschnitten sein, sondern auch f�r andere Fu�ballereignisse tauglich sein. Zus�tzlich kam von unserer Seite die Idee, eine Webanwendung zur Verwaltung der Bundesligaergebnisse. Ein Tippspiel System sollte dann auf diese Daten zur�ckgreifen und so ein Bundesligatippspiel darstellen k�nnen.\\
Dieser Gedanke wurde jedoch aus verschiedenen Gr�nden verworfen. Zum einen war es nicht unserere Idee, sondern die des Javamagazin's und zum anderen wussten wir nicht sofort was bei diesem System alles zu realisieren war. Auch war uns die Funktionsweise dieses Tippspiels nicht sofort klar.
%===========================================
\subsection{Videosoftware} \label{sec:Videosoftware}
Da jeder von uns schon einmal ein Video in einer Videothek ausgeliehen, kam uns der Gedanke einer Onlinevideothek. Solche Videotheken gab es mittlerweile schon wie bspw Amango\footnote{\url{http://www.amango.de}}, Netleih\footnote{\url{http://www.netleih.de}},  Invdeo\footnote{\url{http://www.invdeo.de/}} und Verleihshop \footnote{\url{http://www.verleihshop.de}}. Bei genauer Betrachtung dieser Onlinevideotheken, fragten wir uns wie solch eine Videothek technisch funktioniert. Da wir gerade auf der Suche nach einem idealen Projekt waren, hatten wir damit eins gefunden.\\
Es solle versucht werden eine Online-Videothek mit entsprechenden Modulen zu realisieren.



\section{Zielsetzung und Einsatzbereich} \label{sec:Zielsetzung}
\textbf{Muss noch �berarbeitet und zusammengefasst werden}\\
Zielsetzung dieses Projektes ist dabei Erfahrung mit verschiedenen neuen Technologien zu sammeln und selbst�ndig an einem Projekt zu arbeiten. Sowohl die eigene Gedanken, Ideen, Planung und auch Realisierung dieses Projektes sollten uns auf eine sp�tere Eigenverantwortung im Berufsleben vorbereiten. Das Projekt sollte dabei keine vollst�ndige und fehlerfreie Implementierung darstellen. Uns war bewu�t, dass wir nur einen einfachen Prototypen einzelner Module realisieren k�nnten.\\


Diese Software ist sowohl f�r kleine als auch f�r grosse Unternehmen gedacht. Dabei ist es unwichtig, ob es sich um eine reine OnlineVideothek oder um eine richtige Videothek, die jetzt auch per Versand irhe Videos verleihen m�chte, handelt. Durch weitere Module kann die Software so erweitert werden, dass die Software auch f�r eine richtige Videothek geeignet ist. 
%===========================================
%====================================
\chapter{Konzepte und Aufbau} \label{sec:KonzepteundAufbau}
%===========================================
\section{Aufbau}\label{sec:Aufbau}
\subsection{Beschreibung des Gesamtsystems}\label{sec:ersteGedanken}
Bei einer klassischen Videothek besucht der Kunde das Ladengesch�ft der Videothek und st�bert dabei nach Videos, die er gerne an diesen Abend schauen m�chte. Dabei muss die Videothek eine m�glichst gr��e Ladenfl�che besitzen um die Videos dem Kunden zu pr�sentieren. Nachdem der Kunde sich f�r ein Video entschieden hat, nimmt er entweder die leere Verpackung oder ein Plastikschild mit einer Nummer zum Verleihschalter der Videothek. Nachdem der Kunde seine Kundenkarte vorzeigt und durch sein Passwort oder seine Unterschrift verifiziert wurde, sucht der Mitarbeiter anhand einer Nummer in der Leerverpackung oder des Plastikschildes das entsprechende Video heraus, markiert dieses Video im System und gibt es dem Kunden. Dies ist der klasische Ablauf in einer Videothek.\\
Bei einer Online-Videothek kann der Kunde, durch den Versand der Videos, keine Videos f�r den gleichen Abend ausleihen. Er ist gezwungen, sich einige Tage vorher f�r ein oder mehrere Videos zu entscheiden. Der Ablauf unterscheidet sich von einer klassischen Videothek. Der Kunden "`besucht"' die Webseite der Videothek und sucht im Angebot nach Filmen die er sich ausleihen m�chte. Nachdem die Verf�gbarkeit �berpr�ft wurde, legt er Videos in seinem Warenkorb ab. Durch Eingabe seines Benutzernamens und das zugeh�rige Passwort wird der Kunde verifiziert. In dem Lager der Videothek nimmt ein Mitarbeiter die Bestellung �ber einen Monitor oder eine ausgedruckte Liste entgegeben und bearbeitet die Bestellung. Dabei sucht dieser die Videos f�r den Kunden heraus, nimmt die Videos in das System auf und versendet die Videos zu dem Kunden per Post. Der Kunde erh�lt seine gew�nschten Videos, kann diese sich anschauen und schickt diese nach einer bestimmten Zeit an die Videothek zur�ck.\\
Das hier gew�nschte System soll eine komplette Videothek ersetzen. Die Online-Videothek ben�tigt nur noch ein Lager f�r die zu verleihenden Videos und wenige Mitarbeiter f�r den Versand der Videos und die Verwaltung der Videothek. Die Software wird  dabei wie in \vref{fig:erstegedanken} zu sehen ist, von zwei verschiedenen Personenkreisen benutzt, dem Kunden und dem Mitarbeiter. Der Kunde kann die in der Abbildung dargestellten Aktionen ausf�hren, wie bspw. betrachten und bestellen von Videos. Der Mitarbeiter kann dabei das System verwalten und Bestellungen der Kunden bearbeiten.\\
%===========================================
\begin{figure}[p]
	\rotatebox{90}{
		\includegraphics[scale=0.65]{images/erste-gedanken.jpg}
	}
	\caption{Erste Gedanken zu der Videosoftware}
	\label{fig:erstegedanken}
\end{figure}
%===========================================
Nach einiger �berlegung wurde festgestellt, das die Software aus drei anstatt zwei Modulen bestehen muss, wie in \vref{fig:dreimodule} zu sehen ist. Bei den Mitarbeitern der Online-Videothek muss in Verwaltung und Versand/Lager unterschieden werden, da diese unterschiedlichen Aufgaben von unterschiedlichen Mitarbeitern bearbeitet werden. Das \textbf{Kundenmodul} ist die Internetpr�senz der Videothek und repr�sentiert das Unternehmen nach aussen. Auf dieser dynamischen Webseite kann der Kunde die vorhanden Videos durchst�bern und detaillierte Informationen zu den Videos erhalten. Nach erfolgreicher Anmeldung im System kann der Kunde die Verf�gbarkeit des jeweiligen Videos kontrollieren und auf Wunsch Videos ausleihen. In einem zus�tzlichen Men�punkt kann er seine bestellten Videos betrachten und sich ggf. Rechnungen ausdrucken. Das \textbf{Versandmodul} stellt die ben�tigte Software f�r das Lager und den Versand zur Verf�gung. Mit deren Hilfe kann ein Mitarbeiter der Online-Videothek Videos f�r den Versand vorbereiten. D.h. der Mitarbeiter bekommt eine Liste mit Bestellungen von Kunden (elektronisch oder auf Papier) und arbeitet diese ab. Damit der Mitarbeiter nicht jedes mal die Kundennummer und Nummern der Videos eintippen muss, wird seine Arbeit durch Barcodes und Barcodescanner unterst�tzt. Mit dessen Hilfe markiert er Videos f�r einen bestimmten Kunden und eine bestimmte Bestellung und versendet diese. Dem System teilt der Mitarbeiter dadurch mit, das bestimmte Videos nicht mehr verf�gbar sind und von einem bestimmten Kunden ausgeliehen wurde. Rechnungen, Versandetiketten und eventuelle Lieferscheine werden dabei automatisch mit Hilfe eines Druckers erstellt. Zus�tzlich bietet das Versandmodul die M�glichkeit, zur�ck gekommene Videos der Kunden wieder in das System aufzunehmen. Somit wurde das Video wieder vom Kunden zur�ckgegeben und es kann im System als vorhanden/ausleihbar markiert werden oder gleich an den n�chsten Kunden weitergeschickt werden. Das \textbf{Verwaltungsmodul} hilft den Mitarbeitern in der Verwaltung bei der Organisation der Online-Videothek. Es k�nnen Kundendaten und Rechnungen betrachtet und ggf. gedruckt werden. Das Videosortiment kann bearbeitet und inhaltliche �nderungen (z.B. Sonderangebote) an dem Kundenmodul vorgenommen werden. Weiterhin besteht die M�glichkeit ausf�hrliche Reports \& Statistiken zu erstellen und zu betrachten.\\
%===========================================
\begin{figure}[tbp]
	\centering
	\includegraphics[scale=0.65]{images/dreimodule.pdf}
	\caption{Die drei Module der Software}
	\label{fig:dreimodule}
\end{figure}
%===========================================


%===========================================
\subsection{das Kundenmodul im Detail} \label{sec:Kundenmodul}
Das Kundenmodul ist das wichtigste Element der Anwendung, denn dieses Modul wird vom Kunden verwendet und dieser bestimmt �ber Erfolg oder Misserfolg der Online-Videothek und damit �ber die Anwendung. Das Kundenmodul wird, wie bereits erkl�rt, als Webanwendung entwickelt. Somit muss der Nutzer keine spezielle Software auf seinem Rechner verwenden und kann die Software somit �berall verwenden. Grunds�tzlich wird bei der Anwendung zwischen zwei verschiedenen Kundengruppen unterschieden: dem \emph{Interessenten} und dem \emph{angemeldeten Kunden}. Ein \emph{Interessent} ist ein Internetnutzer, der sich �ber das Angebot der Online-Videothek interessiert, ohne bisher ein Video ausgeliehen zu haben. Ist der Interessent von dem Angebot �berzeugt und m�chte ein Video ausleihen, muss er sich im System registrieren (siehe 1. \vref{fig:usecasekundeninteressenten}). Dazu geh�rt neben einem Benutzernamen und Passwort auch seine vollst�ndige Adresse, f�r den Versand der Videos bzw. Rechnungen, und seine Bankverbindung bzw. Kreditkartendaten, f�r die Bezahlung. Ist der Kunde in das System eingeloggt, ist er ein angemeldeter Kunde. Je nach Kundengruppe pr�sentiert sich die Webseite mit anderen Funktionalit�ten. Zuerst sollen die Funktionen und M�glichkeiten eines Interessenten erkl�rt werden. Der Interessent kann sich auf der Webseite der Online-Videothek sowohl �ber das Angebot an Videos (siehe 2. \vref{fig:usecasekundeninteressenten}), als auch �ber die Online-Videothek informieren. �ber verschiedene Webseiten erh�lt er Einblick in die f�r ihn interessante Funktionsweise der Online-Videothek und allgemeinen Daten, wie Impressum, Kontaktdaten und Allgemeine Gesch�ftsbedingungen (siehe 3. \vref{fig:usecasekundeninteressenten}). �ber eine intelligente Suche, Navigationselementen und den Top-Listen (siehe 5. \vref{fig:usecasekundeninteressenten}) kann er nach Videos suchen bzw. st�bern (siehe 4. \vref{fig:usecasekundeninteressenten}). Die Videos sind kategorisiert bzw. geordnet. Der Interessent kann auch detaillierte Informationen �ber einzelne Filme erhalten (siehe 2. \vref{fig:usecasekundeninteressenten}). Dabei kann er sich �ber die Features des Filmes informieren, Kommentare bzw. Bewertungen lesen und �hnliche Filme betrachten. Die dabei verwendete Suche wird als intelligente Suche bezeichnet, weil diese Suche auch �hnliche bzw. verwandte Filme findet und sowohl im Titel, als auch in der Beschreibung der Filme nach dem Stichwort sucht. Es k�nnte auch realisiert werden, das die Suchergebnisse nach Trefferquoten geordnet werden bzw. das Suchfeld automatisch das Wort nach h�ufigen Suchbegriffen erg�nzt.\\
Ist der User eingeloggt hat er die gleiche Funktionsweise wie ein Interessent, nur mit erweiterten Funktionen. In der Ergebnissliste der Suche oder der Anzeigeliste der Kategorien, welche beide gleich aufgebaut sind, befinden sind neben jedem Video Elemente zum Bestellen (siehe 1. \vref{fig:usecasekundenangemeldet}) und zum �berpr�fen der Verf�gbarkeit (siehe 2. \vref{fig:usecasekundenangemeldet}). Jedes Video, also jeder Film, ist in einer gewissen Anzahl vorhanden. Mit Hilfe der �berpr�fung der Verf�gbarkeit kann der Kunde �berpr�fen, ob noch ein Video dieses Filmes f�r ihn zum Ausleihen vorhanden ist, bzw. wann das n�chste Video wieder verf�gbar ist. Je nach Status der Verf�gbarkeit kann der Kunde das Video in seinem Warenkorb legen und damit ausleihen, oder vormerken lassen. Bei einer Vormerkung wird er entweder per Email informiert, dass das Video vorhanden ist, oder das Video wird schnellstm�glich an den Kunden versendet. Hat der Kunden Videos in seinen Warenkorb abgelegt, kann dieser am Ende seiner Bestellung zur Kasse gehen und somit die Videos ausleihen. Weiterhin hat der Kunde die M�glichkeit eine so genannte \emph{Wunschliste} zu Erstellen (siehe 3. \vref{fig:usecasekundenangemeldet}). Diese Liste kann eine bestimmte Anzahl an Videos anschauen, die der Kunde schauen m�chte. Das System arbeitet diese Liste automatisch ab, indem es dem Kunden eine bestimmte Anzahl an Videos mit einmal zusendet und bei R�cksendung durch den Kunden die n�chsten Filme automatisch an den Kunden sendet. Somit hat der Kunden z.B. die M�glichkeit 50 Filme zu bestimmen, die er gerne anschauen m�chte. Das System sendet ihm immer zwei Filme mit einmal zu, die er sich anschauen kann. Sendet er diese zwei Filme zur�ck, bekommt er die n�chsten zwei Filme, bis die Liste abgearbeitet wurde. Weiterhin bietet das Kundenmodul dem Kunden die M�glichkeiten, �ltere und aktuelle Bestellungen bzw. Vorbestellungen zu betrachten (siehe 4. \vref{fig:usecasekundenangemeldet}), Rechnungen auszudrucken (siehe 5.) und allgemeine Einstellungen an seinem Profil vorzunehmen (siehe 6.). Das Kundenmodul soll dabei einfach und ohne Bedienungsanleitung zu bedienen sein, der Kunden soll sich schnell zu recht finden und schnell zu seinem Ziel, einer Bestellung, kommen.\\
\begin{figure}[tbp]
	\centering
	\includegraphics[scale=0.7]{images/usecase-kundenmodul-normale.pdf}
	\caption{UseCase Kundenmodul Interessenten}
	\label{fig:usecasekundeninteressenten}
\end{figure}
\begin{figure}[tbp]
	\centering
	\includegraphics[scale=0.7]{images/usecase-kundenmodul-reguser.pdf}
	\caption{UseCase Kundenmodul angemeldete Kunden}
	\label{fig:usecasekundenangemeldet}
\end{figure}

%===========================================
\subsection{das Versandmodul im Detail} \label{sec:Versandmodul}
Das Versandmodul ist der Bestandteil der Videothek in der die Arbeit geschieht. Hier werden Videos herrausgesucht, verpackt und an den Kunden versendet. Dabei wird der Mitarbeiter soweit wie M�glich von der Technik unterst�tzt. Bei dieser Technik handelt es sich haupts�chlich um Barcodescanner. Ein Barcodescanner ist ein Leseger�t, das mit Hilfe eines Lasers einen Strichcode auf einer Verpackung oder einem Blatt liest. Dieser Strichcode kann dabei je nach Art und Weise eine Nummer oder einen Wort repr�sentieren. Diese Strichcodes sind aus dem heutigen Alltag nicht wegzudenken, denn sie sind auf jeder Verpackung vorhanden und werden in Superm�rkten als Preisschild verwendet. Dabei hat jedes Produkt eine eindeutige Nummer, zu der mit Hilfe einer Datenbank ein Preis zugeordnet wird. Neben Strichcodes gibt es noch zweidimensionale Codes, die mehr Informationen speichern k�nnen.\\
%%http://www.4mation.com.au/images/softwaredevelopment/products/softwaredevelopment_barcodescanner.jpg
%%http://jbars.sourceforge.net/
%%http://jbarcodebean.sourceforge.net/
\begin{figure}[htbp]
	\centering
	\includegraphics[scale=0.7]{images/Barcodescanner.jpg}
	\caption{Abbildung eines Barcodescanners}
	\label{fig:barcodescanner}
\end{figure}
An einem Zentralen Rechner werden die Bestellungen der Kunden ausgedruckt. Solch ein Ausdruck besteht aus einer Seite, auf der oben zwei Adressetiketten und unten einem Lieferschein vorhanden sind. Das erste Adressetikett ist das f�r den Versand zu dem Kunden, das auf den Umschlag der Bestellung geklebt wird. Das zweite ist dasjenige, welches der Kunde auf dem Umschlag zur R�cksendung der Video klebt. Der Lieferschein ist sowohl f�r den Mitarbeiter als auch f�r den Kunden wichtig. Der Mitarbeiter liest mit Hilfe eines Barcodescanners die Bestellnummer ein. Aus dieser Rechnungsnummer kann der Computer sowohl auf den Kunden als auch auf die Videos schlie�en, die versendet werden sollen. Der Mitarbeiter sucht, mit Hilfe einer weiteren Software den Lagerort der jeweiligen Videos heraus und scannt deren eindeutige Nummer. Damit sind die Videos im System nicht mehr verf�gbar und dem Kunden zugeordnet. Anschlie�end verpackt er die Videos in einen entsprechenden Umschlag und verschickt die Videos mit dem Ausdruck an dem Kunden. Bei diesem Vorgang gibt es mehrere M�glichkeiten wie der Mitarbeiter vorgehen kann. Damit der Mitarbeiter nicht f�r jede Bestellung durch das Lager gehen muss, kann das System eine Liste mit Videos erstellen, f�r die n�chsten 10 Bestellungen. Somit sucht der Mitarbeiter diese Videos heraus und bearbeitet nacheinander diese zehn Bestellungen. Eine weitere M�glichkeit w�re, das der Mitarbeiter und das Lager durch ein automatisiertes Lager ersetzt wird. Dabei �bernimmt ein Roboter in einem Lager die Aufgabe des Suchen und Finden der Videos. Schickt der Kunde die Videos an die Online-Videothek zur�ck, nimmt ein Mitarbeiter die Videos entgegen. Zuerst scannt er dabei die Bestellnummer und die eindeutige Nummer des jeweiligen Videos. Zus�tzlich muss er den Zustand der Videos �berpr�fen und im System eintragen. Anschlie�end sind die Videos wieder im System verf�gbar und k�nnen wieder eingeordnet werden oder an den n�chsten Kunden versendet werden.\\
Das komplette Lager k�nnte durch spezielle Maschinen vollst�ndig automatisiert werden. Diese Maschinen w�rden dann automatisch das Suchen und das Verpacken der Videos �bernehmen. Dabei w�rden wieder Barcodescanner zum Einsatz kommen, um die Bestellnummern und Videonummers automatisch einzulesen. Solch ein System w�rde sich aber nur bei einer sehr grosser Online-Videothek rentieren, da der Anschaffungspreis solcher Maschinen enorm ist.

\begin{figure}[tbp]
	\centering
	\includegraphics[scale=0.7]{images/usecase-versandmodul.pdf}
	\caption{UseCase Versandmodul}
	\label{fig:usecaseversand}
\end{figure}
	 	
%===========================================
\subsection{das Verwaltungsmodul im Detail}  \label{sec:Verwaltungsmodul}
Das Verwaltungsmodul ist f�r die Verwaltung der Online-Videothek in B�ror�umen gedacht. Die Mitarbeiter des Unternehmens haben dabei Einblick in die verschiedenen Daten der Online-Videothek und k�nnen diese, sollten sie die ben�tigten Rechte besitzen, ver�ndern. Unter zu Hilfenahme des Verwaltungsmodul k�nnen neue Videos in System aufgenommen werden. Dazu wird zuerst der jeweilige Film hinzugef�gt und danach die einzelnen Videos dieses Filmes. Dabei werden sich wiederholende Daten wie Darsteller oder Genre in eigenen Elementen gespeichert. Nachdem der Mitarbeiter mit dem Hinzuf�gen von neuen  Videos fertig ist, werden diese automatisch im Kundenmodul verf�gbar sein. Mit Hilfe dieses Moduls k�nnen auch Kundendaten betrachtet oder ver�ndert werden. Somit kann z.B. eine Telefonhotline oder der Kundensupport Fragen der Kunden zu einzelnen Bestellungen beantworten. F�r die Gesch�ftsleitung k�nnen ausf�hrliche Statistiken erstellt werden.\\
Das Verwaltungsmodul stellt somit ...
\begin{figure}[htbp]
	\centering
	\includegraphics[scale=0.7]{images/usecase-verwaltungsmodul.pdf}
	\caption{UseCase Verwaltungsmodul}
	\label{fig:usecaseverwaltung}
\end{figure}



%===========================================
\section{geplante Module und Versionen}
Da dieses Projekt nach ersten �berlegungen und Planungen nicht nur ein kleiner Projekt ist, wurde beschlossen, zuerst das Verwaltungsmodul, dann das Kundenmodul und danach das Versandmodul zu implementieren. Das Verwaltungsmodul ist das Herzst�ck der Online-Videothek. Hier werden die Daten erstellt und verwaltet, die von den beiden anderen Modulen verwendet werden. Das Verwaltungsmodul und das Versandmodul, soll dabei eine Anwendung auf einem beliebigen Rechner innerhalb des Firmennetzwerks sein. Das Kundenmodul hingegen soll eine weltweit verwendbare Webanwendung sein. Mit der Realisierung des Verwaltungsmodul wurde zuerst begonnen. Im Laufe der Entwicklung des Verwaltungsmoduls wurde das Ausma�e des Projektes sichtbar. Der Umfang des Projektes war gr��er als angenommen. Die Einarbeitungszeit in die Technologien, die Koordination der Zusammenarbeit und die gr��e des Projektes f�hrten zu einer sehr langen Entwicklungszeit des Verwaltungsmoduls. Das Kundenmodul wurde dadurch nur teilweise und anschaulich realisiert werden. Auf eine Realisierung des Versandmoduls wurde komplett verzichtet.




%Hier danach nicht mehr schreiben
\label{sec:KonzepteundAufbau-ende}
%====================================
\chapter{Technologien} \label{sec:Technologien}
%\section{Datenbank}
Zum Einsatz soll eine OpenSource Datenbank kommen. Gedanken an eine kommerzielle Datenbank kam aus Gr�nden der Lizenzkosten nicht auf. \\
Zu Auswahl standen mehrere OpenSource Datenbanken. MySql \footnote{\url{http://dev.mysql.com/downloads/mysql/4.0.html}}, SAP DB \footnote{\url{http://dev.mysql.com/downloads/maxdb/7.5.00.html}} , HSQL DB \footnote{\url{http://hsqldb.sourceforge.net}} und Firebird \footnote{\url{http://firebird.sourceforge.net}}.
		\begin{itemize}
			\item \textbf{MySql}
				\begin{itemize}
					\renewcommand{\labelitemii}{+}
					\item sehr verbreitet
					\item einige Erfahrung
					\item gut Dokumentiert \& gro�e Community
					\item 
				\end{itemize}
				
				\begin{itemize}
					\item schlechtes Lizenzmodell
					\item zu bekannt
					\item keine Trigger
					\item meist nur im privat bzw. klein Unternehmer Einsatz
				\end{itemize}
				
			\item \textbf{SAP DB}
				\begin{itemize}
					\renewcommand{\labelitemii}{+}				
					\item 
					\item Datenbank seit mehreren Jahren bei SAP im Einsatz
				\end{itemize}
				
				\begin{itemize}
					\item schlechte Skalierbarkeit, da der Datenbank Speicherbereich im Vorfeld festgelegt werden muss
					\item schlechte Erfahrung
				\end{itemize}
				
			\item \textbf{HSQLDB}
				\begin{itemize}
					\renewcommand{\labelitemii}{+}				
					\item reine JavaDatenbank
					\item sehr klein
					\item kann als reine Speicher Datenbank verwendet werden (Daten nur im Arbeitsspeicher)
					\item kann als Applikations Datenbank verwendet werden (nur eine Applikation benutzt die Datenbank)
				\end{itemize}
				
				\begin{itemize}
					\item nicht f�r gro�e Applikationen geeignet
					\item 
				\end{itemize}			

			\item \textbf{Firebird}
				\begin{itemize}
					\renewcommand{\labelitemii}{+}				
					\item geringe Erfahrung durch Studium
					\item sehr klein
					\item gute grafische Tools
					\item Original Sourcen kommen von Borland
					\item Interbase Datenbank seit mehreren Jahren im Professionelle einsatz
				\end{itemize}
				
				\begin{itemize}
					\item schlechtes Lizenzmodell
					\item 
				\end{itemize}
		
		\end{itemize}
		
Wir haben uns f�r die Firebird Datenbank entschieden, da es keine wirkliche Konkurrenz im Open Source Bereich gibt.\\
HSQL scheidet schon aus, weil es nicht f�r grosse Datenmengen geeignet ist. Bei der SAP DB muss der ben�tigte Speicherplatz der Datenbank vorher bekannt sein, was bei unserem Projekt nicht der Fall ist. MYSQL unterst�tzt keine Triggers und ist zu bekannt, d.h. MySql kann und sollte jeder Informatiker kennen und benutzt haben. \\
Firebird ist f�r uns relativ neu und die Erfahrungen die wir in der Vorlesung "`Datenmanagment 2"' bekommen haben, war sehr positiv. Da diese Datenbank urspr�nglich von Borland kommt, ist diese Datenbank auch nicht so neu, wie viele Denken.\\
Es soll aber schon am Anfang des Projektes bedacht werden, dass die Datenbank zu einem sp�teren Zeitpunkt eventuell mit einer professionelle Datenbank\footnote{z.B. DB2 von IBM} ausgetauscht werden k�nnte. Deswegen muss schon am Anfang eine hohe Abstraktionsebene vorhanden sein, so dass eventuelle Datenbankspezifische Elemente (Klassen) sehr einfach ausgetauscht werden k�nnen.
	
\section{Versionsverwaltung}
Eine Versionsverwaltung 


\url{http://better-scm.berlios.de/comparison/comparison.html}
\citep[Versionsverwaltung]{Wikipedia2005}

\subsection{Concurrent Versions System - CVS}
		
		
\subsection{Subversion}

\begin{figure}[htbp]
	\centering
	\includegraphics[scale=0.65]{images/subersion-architektur.png}
	\caption{Subversion's Architektur \citep[Kap.~1]{Collins-Sussman2005}}
	\label{fig:svnarchitekur}
\end{figure}

\section{Entwicklungsumgebung} \label{sec:tech-Entwicklungsumgebung}
		\subsection{JBuilder}
		\subsection{Netbeans}
		\subsection{Eclipse}























%Hier danach nicht mehr schreiben
\label{sec:tech-Entwicklungsumgebung-ende}
\section{grafischen Benutzerschnittstellen in Java} \label{sec:tech-Benutzerschnittstellen}
		\subsection{Abstract Window Toolkit - AWT}
		\subsection{Swing}
		\subsection{Standard Widget Toolkit - SWT}
		
		
		
\section{Java-Web-Anwendungen} \label{sec:tech-WebAnwendungen}
		\subsection{Java Server Faces - JSF}
		
Nach \citep{Bill2004} ist JavaServer Faces (JSF) ein komponenten-abh�ngiges Web-Framework, mit dessen Hilfe sich Benutzerschnittstellen mit einer Reihe von wiederverwendbaren GUI-Komponenten, z.B. Labels, Buttons, Eingabefelder usw.,einfach erstellen lassen. JSF ist ein erster offizieller Standard von Sun hinsichtlich der Erstellung von UIs\footnote{User Interface (engl. Benutzerschnittstelle)} von Webanwendungen und wurde im Rahmen des Java Community Process (JCP) unter dem JSR 127\footnote{\url{http://www.jcp.org/en/jsr/detail?id=127}} im September 2002 ver�ffentlicht. Im Experten-Gremium sind viele namhafte Firmen wie IBM, BEA, IONA, Novell, Borland, HP, Oracle oder Siemens, was eine gro�e Unterst�tzung seitens der Industrie erwartet.

\begin{figure}[h]
	\centering
	\includegraphics[scale=1]{images/Architektur-JSF.jpg}
	\caption{Architektur von JSF \citep[Bild 1.6]{Bill2004}}
	\label{fig:architecture_jsf}
\end{figure}

Die auf dem bekannten Model View Controller 2(MVC2)-Modell basierende JSF-Technologie besteht aus den folgenden zwei Hauptkomponenten:

\begin{itemize}
	\item JSF API
	\item JSF Tag Libraries
\end{itemize}

Das Pr�sentations-Framework JSF muss die von MVC definierten Komponenten abbilden. Wie in \vref{fig:architecture_jsf} zu sehen ist, wird das Modell durch einfache JavaBeans, aber auch durch EJBs\footnote{Enterprise Java Beans} oder JDOs\footnote{Java Data Objects} abgebildet. Der Controller wird durch Action Handler bzw. Event Listener der jeweiligen UI-Komponenten dargestellt. Im Zentrum des Controllers steht das FacesServlet, welches mit Hilfe der Konfiguration reagieren, agieren und navigieren kann. JSPs und Komponenten sowie deren Renderer, Converter und Validatoren bilden die Views ab. In jeder View, welche meistens durch eine JavaServer Page (JSP) aufgebaut ist, existiert ein entsprechender Component Tree. Dieser beinhaltet alle Komponenten, die in der JSP durch definierte Tags dargestellt werden. Somit hat der Entwickler Zugriff auf alle Komponenten im Laufe des Lebenszyklus der Request-Bearbeitung.\\

Dieser Lebenszyklus wird zu jeder Anfrage an die JSF-Applikation durchlaufen und enth�lt folgende Phasen:
 
\begin{enumerate}
	\item Restore View: Aufbau des Component Tree
	\item Apply Request Values: Die Daten aus dem Request werden den Komponenten zugeordnet
	\item Process Validations: Die Variablen der Komponenten werden validiert
	\item Update Model Values: Die Variablen der Komponenten werden in deren Datenmodellen gespeichert
	\item Invoke Application: Ausf�hrung der Business-Logik
	\item Render Response: Der Component Tree wird aktualisiert und ein Response generiert
\end{enumerate}		

Die Navigation in einer JSF-Applikation wird in einer Konfigurationsdatei definiert. Darin ist f�r jede JSP jeweils eine Navigationsregel festgelegt. Diese Regeln bestehen aus der Aufrufenden Seite sowie verschiedenen Navigationsf�llen. Solche Fallunterscheidungen machen die Navigation abh�ngig von den Ergebnissen der Businesslogik und sorgen f�r die Dynamik der Anwendung.\\

\textbf{Fazit:} Da in der Projektplanungsphase die derzeitige JSF-Spezifikation zwar relativ ausgereift war, aber noch nicht in einer Finalversion vorlag, wurde in der Phase der Implementierung auf dessen Verwendung verzichtet und stattdessen auf Apache Struts zur�ckgegriffen.

\newpage

\subsection{Apache Struts}



























%Hier danach nicht mehr schreiben
\label{sec:tech-WebAnwendungen-ende}
\section{Persistenzschichten in Java} 	\label{sec:tech-Persistenzschichten}
		
\subsection{Java Data Objects} 
		
JDO wurde im Rahmen eines von Sun initiierten JCR entwickelt. Neben Sun beteiligten sich zahlreiche weitere Firmen an der Entwicklung der Spezifikation, darunter Ericcson, IBM, Informix, Oracle, Poet, Rational, SAP, Software AG und Versant. Der Spezifikationsprozess begann Mitte 1999 und im April 2002 wurde der "`Final Release"' der Spezifikation ver�ffentlicht, wobei die gesammelten Erfahrungen aller beteiligten Firmen mit objektorientierten Datenbanken eingeflossen sind.\\
Die JDO-Spezifikation gibt verschiedene APIs (Application Programming Interface)
und ebenso Richtlinien zu Ihrer Implementierung als sogenannte Contracts vor. SUN selbst
stellt nur eine Referenzimplementation bereit. Auf dem Markt existieren aber viele
weitere Implementationen.\\
JDO kann sehr unterschiedliche (Speicher-) Technologien verwenden. Applikationen mit JDO sollen portabel �ber verschiedene Implementationen wie auch �ber verschiedene Speichertechnologien sein. Hierunter fallen insbesondere relationale und objektorientierte Datenbanksysteme, aber auch Dateisysteme, XML und andere. JDO kann sowohl mit Applikationsservern (Application Server) als auch innerhalb gew�hnlicher (standalone) Java Applikationen eingesetzt werden. Welche Speichertechnologien aber tats�chlich unterst�tzt werden, ist implementationsabh�ngig und kann sehr unterschiedlich sein. (siehe \vref{fig:architecture_jdo})

\begin{figure}[h]
	\centering
	%\includegraphics[scale=1]{images/Architektur-JDO.jpg}
	\includegraphics[scale=1]{images/Architektur-JSF.jpg}
	\caption{Architektur von JSF \citep[Bild 2.0]{JDO2003}}
	\label{fig:architecture_jdo}
\end{figure}

Im folgenden werden die nach \citep{Stadtherr2003} wichtigsten Eigenschaften von JDO genannt:

\begin{enumerate}

\item Unabh�ngigkeit von der JDO-Implementierung: Eine f�r JDO vorbereitete Applikation kann mit einer beliebigen JDO-Implementierung betrieben werden.

\item Unabh�ngigkeit vom Datenspeicher: Die API von JDO abstrahiert vollst�ndig von
dem dahinter liegenden Datenspeicher. Dabei kann es sich um eine flache Datei, eine
objektorientierte Datenbank oder eine relationale Datenbank handeln.

\item Transparente und transitive Persistenz: Transparente Persistenz bedeutet, dass diese nicht im Quellcode sichtbar ist. Wenn z.B. mit setX das Attribut X einer persistenten Instanz ver�ndern und anschlie�end die Transaktion beende wird, dann soll das ge�nderte Attribut automatisch in die Datenbank geschrieben werden. Aufgrund transitiver Persistenz wird ein Objekt auch dann persistent, wenn es bei Transaktionsende �ber einen Pfad von Referenzen von einem bereits persistenten Objekt aus erreichbar ist.

\item Klassen-Enhancement: Die JDO-Persistenzschicht einer Applikation wird durch eine
automatische Instrumentierung der Java-Klassen erzeugt. Diese Instrumentierung
wird im Englischen Enhancing genannt. Die automatische Instrumentierung der
Klassen ist der Schl�ssel f�r transparente und transitive Persistenz, da auf diese Art
die notwendigen Mechanismen vor dem Programmierer versteckt werden k�nnen.

\item XML-Metadaten: Die persistenten Klassen m�ssen in einer XML-Datei beschrieben werden. Im einfachsten Fall besteht die Datei nur aus den Namen der persistenten Klassen. Wenn Arrays, Collections oder Maps verwendet werden sollen, m�ssen dazu auch einige wichtige Informationen (z.B. der Typ in der Collection) in dieser Beschreibungsdatei eingetragen werden.

\item JDO Query Language: JDOQL ist eine Querysprache, die eng an die Syntax von
Java angelehnt ist. Sie ist nicht so m�chtig wie SQL, bietet aber die wichtigsten Elemente
einer Datenbank-Querysprache.

\item J2EE Integration: Die JDO-Spezifikation sieht es vor, kompatibel zu den existierenden J2EE-Frameworks zu sein. JDO ist dabei unabh�ngig von den EJB-Konzepten f�r Container Managed Persistence und Bean Managed Persistence.

\end{enumerate}

\textbf{Fazit:} Da in der Projektplanungsphase die damalige Version von JDO nur eine eingeschr�nkte Abfragesprache und eine mangelnde Standardisierung f�r die Anbindung relationaler
Datenbanksysteme besa�, was f�r das Projekt als Nachteil anzusehen war, wurde nach anderen M�glichkeiten der Persistierung gesucht. Hierbei erschien Hibernate als die umfassendste L�sung.

\newpage


\subsection{Hibernate}
		
Hibernate ist ein Open Source-Persistenz-Tool, das basierend auf so genannten Mappings das Bindeglied zwischen JavaBeans und einer Datenbank darstellt. Seit September 2003 geh�rt das Hibernate-Projekt zur JBoss Group und liegt in der aktuellen Version 3.0 kostenlos zum Download\footnote{\url{http://www.hibernate.org}} bereit. Zur Zeit werden 16 Datenbanken unterst�tzt, worunter unter anderem Oracle, DB2, MySQL sowie PostgreSQL z�hlen. Zu den weiteren Besonderheiten z�hlen die Hibernate Query Language, Native SQL Queries sowie Lazy- und Outer-Join Fetching zur Steigerung der Performance. Ferner l�sst sich Hibernate problemlos in alle bekannten J2EE Application Server integrieren.\\
Hibernate stellt den Entwicklern ein umfangreiches Werkzeug f�r die Realisierung einer leistungsf�higen Persistenzschicht zur Verf�gung. Hierbei werden grundlegende Mechanismen f�r das Laden, Speichern, Aktualisieren und L�schen von Java-Objekten, sowie deren Beziehungen, bereitgestellt.\\
Das Abbilden von Java-Objekten auf eine entsprechende Datenbank erfolgt auf einem �u�erst flexiblen Weg, da sich diese Java-Klassen und die entsprechenden Konfigurationsdateien sehr einfach aus einem bestehenden Datenbankschema generieren lassen. Auch der umgekehrte Weg (Top-Down), d.h. die Generierung eines Datenbankschemas aus bestehenden Java-Klassen, l�sst sich einfach realisieren.

\begin{figure}[h]
	\centering
	%\includegraphics[scale=1]{images/Architektur-Hibernate.jpg}
	\includegraphics[scale=1]{images/Architektur-JSF.jpg}
	\caption{Architektur von Hibernate \citep[Bild 2.1]{Bauer2004}}
	\label{fig:architecture_hibernate}
\end{figure}

\vref{fig:architecture_hibernate} stellt die Rollen der wichtigsten Schnittstellen der Business- und Persistenzschicht von Hibernate dar. Dabei agiert die Businessschicht als ein Client der Persistenzschicht. In manchen Anwendungen werden diese beiden Schichten aber auch nicht getrennt dargestellt. Hibernate erm�glicht nach \citep{Bauer2004} auch die Verwendung von bestehenden Java APIs, wie z.B. JDBC, JTA oder JNDI. JDBC bietet abstrakte Funktionalit�ten analog zu relationalen Datenbanken und erlaubt es, fast jede Datenbank �ber einen JDBC Treiber mit Hibernate verwenden zu k�nnen. JNDI und JTA erm�glichen Hibernate die Integration in J2EE Applikationsservern.\\

�ber XML-basierte Mapping-Dateien wird das objektrelationale Abbilden der Java-klassen f�r Hibernate-Anwendungen definiert. Eine solche Mapping-Datei wird mit dem Dateinamen-Suffix \textit{.hbm.xml} versehen und wird generell f�r jede persistente Klasse erzeugt. In \vref{code:mapping_hibernate} wird das Prinzip der Mapping-Dateien dargestellt.

\begin{lstlisting}[language=XML, caption={Mapping-Datei von Hibernate}, label=code:mapping_hibernate, showstringspaces=false]
<hibernate-mapping>
  <class name="de.hsharz.provirent.objects.Bill" table="BILL">
	
    <id name="billId" type="int" column="BILLID">
      <meta attribute="scope-set">public</meta>
      <meta attribute="use-in-equals">true</meta>
      <generator class="native"/>
    </id>

    <many-to-one name="customer" class="de.hsharz.provirent.objects.Customer">
      <meta attribute="use-in-tostring">true</meta>
    </many-to-one>

    <property name="pdfFile" type="binary">
      <column name="pdffile" sql-type="BLOB" />
    </property>
				
    <property name="pdfFileSize" type="int">
      <meta attribute="use-in-tostring">true</meta>      	
    </property> 

  </class>
</hibernate-mapping>
\end{lstlisting}

In den Mapping-Dateien wird die Zuordnung der einzelnen Attribute (Properties) zu den entsprechenden Tabellenspalten der zugrunde liegenden Datenbank und auch Beziehungen zu anderen persistenten Java-Klassen (Relationen) festgelegt. Die folgende XML-Elemente sollten dabei verwendet werden:

\begin{itemize}

 \item class: Name der Java-Klasse und deren Zuordnung zur korrespondierenden Tabelle der Datenbank
 \item id: Attribut(e) der Klasse f�r den Prim�rschl�ssel
 \item property: Zuordnung der einzelnen Spalten der Datenbanktabelle zu den Properties der Java-Klasse mit zus�tzlichen Angaben �ber den zu mappenden Datentyp und das Erlauben von Null-Werten
 \item many-to-one: Darstellung einer n:1 Beziehung mit Zuordnung der Spalte aus der Datenbanktabelle zu einer entsprechenden Property und Angabe des Objekttyps der Beziehung

\end{itemize}

Au�er diesen gibt es noch weitere Attribute. �ber deren Bedeutung informieren die Hibernate-Webseiten.\\
Standardm��ig wird Hibernate �ber ein zentrales XML-Dokument konfiguriert. Der Name dieser Datei wird meist mit der Endung .cfg.xml gebildet. Darin werden solche Konfigurationen wie Deklaration der Datenbankverbindung, des Hibernate-Dialektes (abh�ngig vom verwendeten DBMS), sowie zus�tzlichen Optionen festgelegt. Ferner k�nnen auch die Ressourcen der einzelnen Mapping-Dateien angegeben werden, um diese der Java-Applikation bekanntzumachen.\\
Vor der Verwendung von Hibernate als Persistenzmechanismus in einer Anwendung muss die Hibernate-Umgebung initialisiert werden. Hierbei wird die Klasse \textit{SessionFactory} in der Gesch�ftslogik der Anwendung geladen. Mit Hilfe dieser Klasse l�sst sich eine Session-Instanz erzeugen, die als ein Bindeglied zwischen der Datenbank und der Anwendung fungiert. Nur �ber diese Session ist die Interaktion mit den Datenbankobjekten m�glich. Darunter sind die so genannten "`CRUD-Methoden"' (create, retrieve, update, delete) oder Queries (Abfragen mit HQL\footnote{Hibernate Query Language}) zu verstehen. \vref{code:save_hibernate} illustriert das Speichern des persistenten Objekts \textit{Dvd} in die Datenbank.\\

\begin{lstlisting}[language=Java, caption={Save-Methode der Hibernate-API}, label=code:save_hibernate, showstringspaces=false]
try {
  Dvd dvd = (Dvd) session.save(new Dvd());
} catch (HibernateException e) {
  logger.error("Objekt konnte nicht gespeichert werden", e);
}
\end{lstlisting}

Hibernate besitzt auch eine Transaktionsschnittstelle. So lassen sich �ber eine Transaktions-Instanz, die �ber das Session-Objekt erzeugt werden kann, Transaktionen durch geeignete Methoden (\textit{begin}/\textit{commit}/\textit{rollback}) abgrenzen. Diese Schnittstelle ist insofern erweiterbar, dass sie leicht mit anderen Systemen integriert werden kann. Weiterhin stehen dem Entwickler als Transaktionsstrategien sowohl optimistisches als auch pessimistisches Locking  zur Verf�gung.\\
Ein weiteres Merkmal von Hibernate ist die M�glichkeit der automatischen Generierung von Prim�rschl�sseln, wobei an die 10 verschiedenen M�glichkeiten, wie z.B. Sequenzen, im Vordergrund stehen.\\
Dar�ber hinaus stellt Hibernate verschiedene Abfragesprachen zur Verf�gung. Hierbei sind die Hibernate Query Language (HQL), Query By Criteria und Query By Example zu nennen. HQL ist an SQL angelehnt, beherrscht aber auch objektorientierte Konzepte wie Vererbung und Assoziationen. Die Anfragen werden dabei in Zeichenketten abgelegt und Hibernate �bergeben. Dieses Konzept l�sst sich in den Referenz-Dokumenten\footnote{siehe \citep{Hibernate2005}} von Hibernate  genauer betrachten. Bei der Verwendung von Query By Criteria werden keine Zeichenketten benutzt, sondern eine Anfrage setzt sich aus einzelnen Ausdr�cken zusammen, die zu einer so genannten \textit{CriteriaQuery} hinzugef�gt werden. Demnach wird die Syntax der Abfragen bereits zur �bersetzungszeit durch den Compiler und nicht erst zur Laufzeit �berpr�ft. Die Query By Example Schnittstelle nutzt das Konzept von Query By Criteria. Hierbei wird eine mit entsprechenden Suchdaten versehene Beispielklasse einer \textit{CriteriaQuery} �bergeben, woraufhin diese alle Klassen zur�ckliefert, die den Eigenschaften der �bergebenen Klasse entsprechen.

























%Hier danach nicht mehr schreiben
\label{sec:tech-Persistenzschichten-ende}
%====================================
\chapter{Implementierung} \label{sec:Implementierung}


\section{Versionsverwaltung mit Subversion} \label{sec:impl-Versionsverwaltung}
Als Versionsverwaltung wurde Subversion verwendet. Dies hat mehrere Gr�nde: Zum einen war Subversion zu diesem Zeitpunkt eine neue und z.T. auch unbekannte Technologie, die somit ihren Reiz hatte. Zum anderen wurde Subversion als Nachfolger von CVS angepriesen und sollte viele Nachteile eliminieren. Eins der wichtigen Vorteile von Subversion ist die geringe Netzwerklast. Bei einem "`Commit"' werden nur die Unterschiede zur Vorg�ngerversion �bertragen und nicht wie bei CVS jede Datei komplett. Um Subversion verwenden zu k�nnen, wurde ein Rechner mit einem installieren Subversion Server ben�tigt, der m�glichst 24h im Internet verf�gbar ist. Um mit einem Client auf einen Subversion Server zuzugreifen, gibt es verschiedene �bertragungsprotokolle: Ein eigenes Subversion protokoll (svn), eine Kombination aus Secure Shell und dem eigenen Protokoll (ssh+svn), �ber http oder �ber https. Die sicherste Methode �ber ssh+svn, jedoch erfordert diese die Installation eines SSH Server, was unter Linux, dank verschiedener Anleitungen, einfach geht, jedoch unter Windows nicht so einfach zu realisieren ist. Unter Windows wurde zu Testzwecken ein Apache2 Webserver mit einem Zertifikat zur �bertragung von Daten per https installiert. Dieser Webserver wurde mit dem Subversion Server kombiniert und jeglicher Datentransfer f�r Subversion erfolgte �ber https und dessen eingestellen Port. Der Nachteil war jedoch, dass keiner die M�glichkeit hatte, einen PC 24h online zur Verf�gung zu stellen. So wurde nach einiger Suche im Internet der Open Source Anbieter berlios.de entdeckt. Dort konnten Open-Source Projekte ihre Quellcodes in einer Subversion Versionsverwaltung mittels ssh+svn kostenlos hosten.\\





\texttt{\textbf{maximal zwei Seiten}}


Befehl zum Erzeugen eines neues Repository.
Zugangsbeschr�nkungen
Clientauswahl und Verwendung mit Screenshoots.
Typischer Aufbau eines Root-Verzeichnisses (trink,branches,tags)
Tags erstellen
mit TortoiseSVn nur ein bestimmtes Verzeichniss auschecken, sichbar machen.
automatischer Versand von Emails, nach dem Commit.







%Hier danach nicht mehr schreiben
\label{sec:impl-Versionsverwaltung-ende}
\section{Entwicklungsumgebung mit Eclipse} \label{sec:impl-Entwicklungsumgebung}



















%Hier danach nicht mehr schreiben
\label{sec:impl-Entwicklungsumgebung-ende}
\section{grafischen Benutzerschnittstellen mit SWT} \label{sec:impl-Benutzerschnittstellen}




















%Hier danach nicht mehr schreiben
\label{sec:impl-Benutzerschnittstellen-ende}
\section{Java-Web-Anwendungen mit Struts} \label{sec:impl-WebAnwendungen}



\footnote{So wird eine Fussnote gemacht}\\
% das ist ein Kommentar
\url{http:/www.phil-schneider.de}
\citep{Frotscher2004b} Verweis auf eine Literaturquelle
\emph{hervorgehoben}\\
\textbf{Fett}\\
\texttt{Typewriter}\\
















%Hier danach nicht mehr schreiben
\label{sec:impl-WebAnwendungen-ende}
\section{Persistenzschicht mit Hibernate} \label{sec:impl-Persistenzschichten}

Die Implementation der Persistenzschicht unter Verwendung von Hibernate wurde in mehreren Schritten vollzogen:

\begin{enumerate}

\item Erzeugen der Hibernate-Mapping-Dateien
\item Generieren der Mapping-Klassen aus den XML-Dateien
\item Generieren des Datenbankschemas
\item Erzeugen der Hibernate-Konfiguration
\item Entwicklung der Klasse \textit{Database} f�r die Interaktion mit der Datenbank

\end{enumerate}

Diese Schritte werden im Folgenden etwas genauer beschrieben.

\subsection{Erzeugen der Hibernate-Mapping-Dateien}

Wie in Kapitel \ref{sec:tech-hibernate} beschrieben, sind die Mapping-Dateien f�r das Abbilden eines Datenbankobjekts auf eine Java-Klasse notwendig.\\
Zun�chst wurde in der Phase der Implementierung ein entsprechendes Datenmodell mit allen ben�tigten Entit�ten und Beziehungen erzeugt. Daraus wurden die entsprechenden Hibernate-Mapping-Dateien gebildet, indem hier unter Angabe von Name, Typ, Name der korrespondierenden Tabellenspalte der Datenbank und einigen anderen zum Teil auch optionalen Attributen alle Properties eines persistenten Objekts definiert wurden.
Wichtig ist auch, dass neben den Properties mit einfachen Datentypen auch s�mtliche Beziehungen zu anderen Objekten bekannt gemacht werden mu�ten.

\subsection{Generieren der Mapping-Klassen aus den XML-Dateien}

Nachdem alle Mapping-Dateien erzeugt wurden, war es notwendig daraus die entsprechenden Java-Klassen erzeugen zu lassen. Hierf�r wurde das Tool zur Code-Generierung von Hibernate verwendet. Die Automatisierung dieses Prozesses wurde in ein Ant Build-Skript eingef�gt, indem das Tool mit den entsprechenden Optionen von diesem Skript aus ausgef�hrt werden kann.\\
Wie in \vref{code:generate_classes} ersichtlich, wird �ber das Element \textit{taskdef} das Tool in das Skript integriert und �ber \textit{target} die Anweisungen zur Ausf�hrung definiert.
Mithilfe des Ant-Plugins f�r Eclipse konnte dieses Skript zur Ausf�hrung gebracht werden.

\begin{lstlisting}[language=XML, caption={Erzeugung der persistenten Java-Klassen �ber Apache Ant}, label=code:generate_classes, showstringspaces=false]
    <!-- Teach Ant how to use Hibernate's code generation tool -->
    <taskdef name="hbm2java"
             classname="net.sf.hibernate.tool.hbm2java.Hbm2JavaTask"
             classpathref="project.class.path"/>

    <!-- Generate the java code for all mapping files in our source tree -->
    <target name="codegen"
             description="Generate Java source from the O/R mapping files">
        <hbm2java output="${source.root}">
          <fileset dir="${source.root}">
            <include name="**/*.hbm.xml"/>
          </fileset>
        </hbm2java>
    </target>
\end{lstlisting}

\subsection{Generieren des Datenbankschemas}

F�r die Verwaltung der Provirent-Datenbank wurde die Verwendung des DBMS Firebird vorgezogen.
Dabei wurde ein entsprechender Server auf den Entwicklungsrechnern installiert. Mithilfe des Firebird-Clients FlameRobin war der projekt-externe Zugriff auf diesen Datenbankserver und dementsprechend auf die Datenbank m�glich.\\
Zuvor musste jedoch erst das Datenbankschema generiert werden. Hierbei wurde das Hibernate-Tool zur Generierung von Datenbankschemata verwendet. Wie bei der Generierung der persistenten Java-Klassen wurden die Anweisungen f�r die Generierung in ein Ant Build-Skript, wie in \vref{code:generate_schema} zu sehen, integriert.

\begin{lstlisting}[language=XML, caption={Erzeugung des Datenbankschemas �ber Apache Ant}, label=code:generate_schema, showstringspaces=false]
    <!-- Generate the schemas for all mapping files in our class tree -->
    <target name="schema" depends="compile"
            description="Generate DB schema from the O/R mapping files">

      <!-- Teach Ant how to use Hibernate's schema generation tool -->
      <taskdef name="schemaexport"
               classname="net.sf.hibernate.tool.hbm2ddl.SchemaExportTask"
               classpathref="project.class.path"/>

      <schemaexport properties="\${class.root}/hibernate.properties" quiet="no" text="no" drop="no" delimiter=";">
        <fileset dir="\${class.root}">
          <include name="**/*.hbm.xml"/>
        </fileset>
      </schemaexport>
    </target>
\end{lstlisting}

\subsection{Erzeugen der Hibernate-Konfiguration}

Die Konfiguration des Hibernate-Frameworks bedurfte es, zum einen die Eigenschaften der Datenbankverbindung zu deklarieren und zum anderen die verschiedenen Mapping-Dateien dem Framework bekannt zu machen.\\
Die Datenbankverbindung wurde in einer Properties-Datei (siehe \vref{code:hibernate_properties}) beschrieben. Dabei war es unter anderem notwendig, folgende Eigenschaften zu definieren:

\begin{itemize}

\item Dialekt der Datenbank
\item Datenbanktreiber
\item URL der Datenbank
\item Benutzername und Passwort f�r die DB

\end{itemize}

\begin{lstlisting}[language=XML, caption={hibernate.properties}, label=code:hibernate_properties, showstringspaces=false]

hibernate.dialect=net.sf.hibernate.dialect.FirebirdDialect
hibernate.connection.driver_class=org.firebirdsql.jdbc.FBDriver
hibernate.connection.url=jdbc:firebirdsql:localhost:c:/video
hibernate.connection.username=SYSDBA
hibernate.connection.password=masterkey
hibernate.jdbc.use_streams_for_binary=true

\end{lstlisting}

Zus�tzlich dazu musste noch eine Klasse implementiert werden, die Sessions f�r das Hibernate-Framework bereitstellt. Mithilfe der Klasse \textit{HibernateUtil} wurden nicht nur Methoden zum Erzeugen und Beenden von Sessions angeboten, sondern auch das objektrelationale Mapping der Provirent-Datenbank f�r Hibernate sichtbar gemacht. Hierbei wird eine SessionFactory erzeugt, dessen Konfiguration die Namen aller persistenten Klassen besitzt und somit Zugriff darauf hat.

\subsection{Entwicklung der Klasse \textit{Database} f�r die Interaktion mit der Datenbank}

Nachdem nun das Hibernate-Framework alle entsprechenden Einstellungen erhalten hat, war es m�glich dar�ber mit der Datenbank zu interagieren. Um jedoch eine zentrale Klasse mit allen Operationen zu erhalten, wurde die Klasse \textit{Database} implementiert. Hier befinden sich z.B. Methoden zum Laden, Speichern, Aktualisieren oder L�schen von einzelnen oder mehreren Datenbankobjekten.\\
Die Methode \textit{getSingleActor}() in \vref{code:method_getSingleActor} l�dt einen Schauspieler anhand seiner ID aus der Datenbank und gibt diesen als Objekt der Klasse \textit{Actor} zur�ck.

\begin{lstlisting}[language=Java, caption={Methode zum Laden eines Schauspielers aus der DB}, label=code:method_getSingleActor, showstringspaces=false]
public static Actor getSingleActor(final int id) {
	if (logger.isDebugEnabled()) {
		logger.debug("getSingleActor() - start. int filter= " + id);
	}
	//init the returnlist
	Actor returnobject = null;

	Session s = null;
	Transaction tx = null;
	try {
		//get new Session and begin Transaction
		s = HibernateUtil.currentSession();

		returnobject = (Actor) s.get(Actor.class, new Integer(id));

	} catch (Exception e) {
		logger.error("getSingleActor() - Error while trying to do Transaction", e);

	} finally {
		try {
			// No matter what, close the session
			HibernateUtil.closeSession();
		} catch (HibernateException e1) {
			logger.error("getSingleActor() - Could not Close the Session", e1);
		}
	}

	if (logger.isDebugEnabled()) {
		logger.debug("getSingleActor() - end");
	}
	return returnobject;
}
\end{lstlisting}














%Hier danach nicht mehr schreiben
\label{sec:impl-Persistenzschichten-ende}
%====================================
\chapter{Fazit} \label{sec:Fazit}
Mit Hilfe dieses Projektes wurde bei allen Projektmitglieder Erfahrungen im Bereich Softwareplanung und Softwareentwicklung gesammelt. Bei diesem Projekt wurde eine eigenst�ndige Idee in die Realit�t umgesetzt, wobei alle Projektmitglieder dabei intensiv und mit viel Interesse an einer m�glichst guten Realisierung mitarbeiteten.\\

Technologisch gesehen haben sich die Projektmitglieder in Themengebiete gewagt, die ihnen bis dahin zum Teil relativ fremd waren. So wurden im Bereich vom objektrelationalen Mapping �berhaupt die ersten Erfahrungen gesammelt. Dazu wurde die Erkenntnis gewonnen, dass Hibernate ein sehr vorteilhaftes und umfassendes Werkzeug ist, um die Persistenzschicht von der Businessschicht unabh�ngig zu gestalten.\\
So wurde erreicht, dass diese Persistenzschicht vollst�ndig implementiert werden konnte und f�r den Zugriff aus dem Kundenmodul und dem Verwaltungsmodul zur Verf�gung steht.\\

Weiterhin ist festzustellen, dass die Implementierung des Projekts noch nicht abgeschlossen ist, so dass z.B. noch grundlegende Funktionalit�ten des Kundenmoduls umgesetzt werden m�ssen. Trotzdem konnte und kann hier die Verwendung des Webframeworks Apache Struts dazu beitragen, dass mit relativ wenig Aufwand und Entwicklungszeit die Entwicklung einer 3-schichtigen Webanwendung erm�glicht wird. Dies hat den Vorteil, dass Design, Businesslogik und Daten von einander getrennt dargestellt und umgesetzt werden konnten.\\




Durch die Verwendung von Subversion als Versionsverwaltung wurde der Umgang mit einer neuen Technologie schnell zur Gewohnheit. Dadurch ist es f�r die Projektmitglieder zum Alltag geworden, �nderungen sofort f�r alle zu speichern und mit aussagekr�ftigen Kommentaren zu versehen. Durch die netzwerksparende �bertragung von Daten, war auch der Einsatz von Modems m�glich. Durch den  automatischen Versand von Emails wurden andere Projektmitglieder schnell und kompakt �ber �nderungen informiert. Die installation auf einem Windowstestrechner mit einer gesicherter HTTPS-Verbindung gestaltete sich als unkompliziert, wie die Installation einer gesicherten ssh+svn Verbindung auf einem Fedore Core 3 Linux Rechner, dank gut beschriebener Anleitungen.\\

Aufgrund der SWT-Bibliothek konnte eine einfach zu bedienende Benutzeroberfl�che erstellt werden, mit der alle notwendigen Objekte f�r eine Online-Videothek einfach editiert und erstellt werden k�nnen. Aufgrund der verwendeten Reiter f�r jedes Untermen� kann schnell zwischen den einzelnen Men�s umgeschaltet und Ver�nderungen vorgenommen werden. Die verschiedenen Dialogfenster erleichtern die Eingabe f�r den Benutzer und zugleich die Bearbeitung im Quellcode.\\

Dieses Projekt k�nnte auch als Grundlage f�r andere Studentenprojekte dienen. Dabei m��en die Studenten sich nicht in dieses spezielle Projekt einarbeiten, sondern k�nnen ein selbst�ndiges Modul entwickeln. So k�nnten Studenten bspw. ein Suchmodul mit Apache Lucence entwickeln oder ein Barcodemodul zur Verarbeitung von Barcodes entwickeln. Weiterhin k�nnte ein Modul zur Erstellung von Rechnungen oder anderen Reports mit Eclipse Birt erstellt werden, ein Modul zur Verwaltung der Standorte/Lagerorte der DVD's oder ein Modul zur Berechnung der k�rzesten Wegstrecke bei der Zusammenstellung einer bzw. mehrerer Bestellungen. Jedes dieser Module k�nnte dann mit geringen Aufwand in die Online-Videothek integriert werden.



%====================================
\chapter{Zusammenfassung} \label{sec:Zusamenfassung}


\textbf{\emph{Hier muss beschrieben werden, was in den einzelnen Abschnitten beschrieben wurde.}}
%====================================


\appendix
\chapter{Protokoll vom 11. Mai 2004}

Drei Frameworks stehen zur Auswahl \\
\begin{itemize}
	\item Apache Cocoon
	\item Apache Struts
	\item Apache Tapestry
\end{itemize}

Jeder erstellt eine einfache (Web)Anwendung mit Hilfe eines dieser Frameworks 
folgende Komponenten sollen/muessen enthalten sein: \\
\begin{itemize}
	\item einfache LoginSeite (�ber Datenbank)
	\item Liste aller Videos in Datenbank anzeigen
	\item EingabeMaske f�r neues Labor
	\item Validierung der Eingabedaten
	\item dynamische Navigation
	\item eventuell ein Bild f�r den Status der einzelnen Bilder (dynamisches Bild??)
\end{itemize}

F�r diese BeispielAnwendung sollen m�glichst viele Elemente des jeweiligen Framework verwendet werden.
Wichtig ist dabei der Umgang und die Bedienbarkeit des Systems. \\
Wiederverwendbarkeit einzelner Module. \\
Design und Logik Trennung vorhanden? Kann das Design einfach/schnell ausgetauscht werden.\\
\\
Es geht dabei nicht um ein 100% fehlerfreies System.\\
Design spielt keine wichtige Rolle, es sollte jedoch beachtet werden, dass dieses sp�ter vom Kunden ausgetauscht werden m�chte.\\
\\
\\
\begin{itemize}
	\item Struts: Stefan
	\item Cocoon: Philipp
	\item Tapestry: Remo
\\
	\item Namen f�r das Projekt finden
	\item Link mit Beispiel Webseiten rumschicken
\end{itemize}
	
%%%%%%%%%%%%%%%%%%%%%%%%%%%%%%%%%%%%%%%%%%%%%%%%%%%%%%%%%%%%%%%%%
% 																															%
%---------------------------------------------------------------%
%                         9-1Literatur.tex                      %
%%%%%%%%%%%%%%%%%%%%%%%%%%%%%%%%%%%%%%%%%%%%%%%%%%%%%%%%%%%%%%%%%

%\nocite{Wessel2005}
\addcontentsline{toc}{chapter}{Literaturverzeichnis}
\bibliography{Provirent-Doku}
%%%%%%%%%%%%%%%%%%%%%%%%%%%%%%%%%%%%%%%%%%%%%%%%%%%%%%%%%%%%%%%%%%
% 																															%
%---------------------------------------------------------------%
%            9-2Abkuerzungsverzeichnis.tex                      %
% Abk�rzungen sp�ter in den Text direkt uebernehmen, dort wie diese das erste mal verwendet werden, somit 
% kann mittels refpage auf die Seite verwiesen werden
%%%%%%%%%%%%%%%%%%%%%%%%%%%%%%%%%%%%%%%%%%%%%%%%%%%%%%%%%%%%%%%%%
\addcontentsline{toc}{chapter}{Abk�rzungsverzeichnis}


\printnomenclature
%\addcontentsline{toc}{chapter}{Index}
%\printindex
%%%%%%%%%%%%%%%%%%%%%%%%%%%%%%%%%%%%%%%%%%%%%%%%%%%%%%%%%%%%%%%%%%
% 																															%
%---------------------------------------------------------------%
%                        9-2Erklaerung.tex                      %
%%%%%%%%%%%%%%%%%%%%%%%%%%%%%%%%%%%%%%%%%%%%%%%%%%%%%%%%%%%%%%%%%
\addcontentsline{toc}{chapter}{Erkl�rung}
\chapter*{Erkl�rung}

\vspace*{2cm}
Ich erkl�re hiermit, dass ich zur Anfertigung der vorliegenden Arbeit keine anderen als die angegebenen Quellen und Hilfsmittel und keine nichtgenannte fremde Hilfe in Anspruch genommen habe. Mir ist bewusst, dass eine falsche Versicherung rechtliche Konsequenzen hat.

\vspace{4cm} Leipzig, den \today  \\





%------ Ende des Dokumentes ------
\end{document}