

\chapter{Einf�hrung} \label{sec:Einfuehrung}



\section{Grundlegendes} \label{sec:Grundlegendes}

\subsection{Betreuender Professor}
%===========================================
Hochschule Harz \\
Prof. Dr. Sigurd G�nther \\
Friedrichstr. 57- 59 \\
38855 Wernigerode \\
sguenther@hs-harz.de
%===========================================
\subsection{Studenten}
%===========================================		
\begin{tabular}{rrr}
	Philipp Schneider 				& Stefan Forstner 				& Remo Griesch  \\
	Kastanienring 16					&	Strasse der Jugend 22  	& Strasse				\\
	04316 Leipzig							&	04880 Dommitsch					& Ort						\\[0.4cm]
	pschneider@profi-sell.de	&	sforstner@profi-sell.de	& rgriesch@profi-sell.de \\
\end{tabular}		
			


\section{Motivation} \label{sec:Motivation}
Im Rahmen des Studiums an der Fachhochschule Harz in Wernigerode mu� jeder Student des Studiengangs Kommunikationsinformatik eine Projektarbeit abgeben. Dies bedeutet, da� der Student eine Aufgabe (meist Programmieraufgabe) alleine oder in kleinen Teams bew�ltigen muss. Von Seiten der Hochschule und der Dozenten werden sehr viele interessante Projektarbeiten angeboten, diese Projektarbeiten sind jedoch meist sehr an ein bestimmtes Thema gebunden. Dem Studenten steht es aber auch frei, seine eigene Projektarbeit einzubringen.\\
Da wir schon im Teamprojekt\footnote{Auch das Teamprojekt ist Bestandteil des Studiums. Beim Teamprojekt m��en mehrere Studenten (7-15) gemeinsam eine Programmieraufgabe umsetzen. In diesem Fall ist die Rede vom Teamprojekt Labmin \url{http://labmin.de.vu }}  eine Interessante Aufgabe hatten, wollten wir dieses dort gelernte Wissen aufgreifen und vertiefen. Jedoch wollten wir nicht nur eine einfache Webapplikation entwickeln, denn dies w�rde keine wirkliche Herrausfordung stellen. Da Java mit seiner Enterprise Edition eine gute Programmiersprache f�r gro�e SoftwareSysteme ist, m�chten wir mit diesem Projekt erste Erfahrungen mit J2EE Anwendungen und Servern sammeln, damit wir diese sp�ter in unserem Job anwenden k�nnen.


\section{Ideen} \label{sec:Ideen}
\subsection{Tippspiel}
Die erste Idee wasr die Umsetzung eines Tippspiels in Java, passend zu Europameisterschaft in Portugal. Diese Idee wurde im JavaMagazin in mehreren Ausgaben aufgegriffen und verschiedene Ansatzm�glichkeiten diskutiert. Dieses Tippspiel sollte grob aus einer Datenbank und einer Webseite bestehen. Spieler sollten sich in verschiedenen Tippgemeinschaften, mit je einem Tippgemeinschaftverwalter, zusammen tun und gemeinsam die EM 2004 tippen. Das Tippspiel sollte jedoch nicht genau auf die EM 2004 zugeschnitten sein, sondern auch f�r andere Fu�ballereignisse tauglich sein.\\
Zus�tzlich kam von unserer Seite die Idee, eine Webanwendung zur Verwaltung der Bundesligaergebnisse. Ein Tippspiel System sollte dann auf diese Daten zur�ckgreifen und so ein Bundesligatippspiel darstellen.\\
Dieser Gedanke wurde jedoch aus verschiedenen Gr�nden verworfen. Zum einen war es nicht unserer Gedanke, sondern der des Javamagazin's und zum anderen wussten wir nicht ob diese Aufgabe f�r drei Studenten ausreichen w�rde.
%===========================================
\subsection{VideoSoftware}
Jeder von uns hatte schon einmal ein Video in einer Videothek ausgeliehen. Solche Videotheken gibt es mittlerweile auch als reine Online-Videotheken.\footnote{ siehe \url{http://www.amango.de}, \url{http://www.netleih.de} und \url{http://www.verleihshop.de}}  
Als wir uns bei einer solchen Online-Videothek ein Video ausliehen, fragten wir uns wie solch eine Videothek technisch funktionieren w�rde. Da wir gerade auf der Suche nach einem idealen Projekt waren, hatten wir dies damit gefunden.\\
Es solle also eine Online-Videothek mit entsprechenden Modulen f�r den Kunden, die Verwaltung und das Lager/Versand entstehen. 



\section{Zielsetzung} \label{sec:Zielsetzung}
Zielsetzung dieses Projektes ist es eine einfache Version des Gesamtsystems zu programmieren und in einem stabilen Zustand auf einem Testsystem zu testen. Diese einfache Version soll die Grundfunktionalit�t des Systems darstellen und durch Updates auf eine komplexe Version erweiterbar sein.

