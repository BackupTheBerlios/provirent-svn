\chapter{Einf�hrung} \label{sec:Einfuehrung}



\section{Grundlegendes} \label{sec:Grundlegendes}



\subsection{Betreuender Professor}
%===========================================
Hochschule Harz \\
Prof. Dr. Sigurd G�nther \\
Friedrichstr. 57- 59 \\
38855 Wernigerode \\
sguenther@hs-harz.de
%===========================================
\subsection{Studenten}
%===========================================		
\begin{tabular}{rrr}
	Remo Griesch							& Stefan Forstner 				& Philipp Schneider 				\\
	E.Th�lmannstr.248					&	Strasse der Jugend 22  	& Kastanienring 16				\\
	06528 Blankenheim					&	04880 Dommitsch					& 04316 Leipzig							\\[0.4cm]
	Romeodied@gmx.de					&	fossiossi@web.de				& provirent@phil-schneider.de \\
	
	
	
\end{tabular}		
			
\subsection{Kapitelarbeit} \label{sec:Kapitel}

\subsubsection{Remo Griesch}
\ref{sec:tech-Entwicklungsumgebung} \vpagerefrange{sec:tech-Entwicklungsumgebung}{sec:tech-Entwicklungsumgebung-ende};

\ref{sec:tech-Benutzerschnittstellen} \vpagerefrange{sec:tech-Benutzerschnittstellen}{sec:tech-Benutzerschnittstellen-ende};

%\ref{sec:impl-Entwicklungsumgebung}  \vpagerefrange{sec:impl-Entwicklungsumgebung}{sec:impl-Entwicklungsumgebung-ende};

\ref{sec:impl-Benutzerschnittstellen}  \vpagerefrange{sec:impl-Benutzerschnittstellen}{sec:impl-Benutzerschnittstellen-ende}

\subsubsection{Stefan Forstner}
\ref{sec:tech-WebAnwendungen} \vpagerefrange{sec:tech-WebAnwendungen}{sec:tech-WebAnwendungen-ende};

\ref{sec:tech-Persistenzschichten}  \vpagerefrange{sec:tech-Persistenzschichten}{sec:tech-Persistenzschichten-ende};

\ref{sec:impl-WebAnwendungen} \vpagerefrange{sec:impl-WebAnwendungen}{sec:impl-WebAnwendungen-ende};

\ref{sec:impl-Persistenzschichten}  \vpagerefrange{sec:impl-Persistenzschichten}{sec:impl-Persistenzschichten-ende}

\subsubsection{Philipp Schneider}
\ref{sec:Einfuehrung} \vpagerefrange{sec:Einfuehrung}{sec:Einfuehrung-ende};

\ref{sec:KonzepteundAufbau} \vpagerefrange{sec:KonzepteundAufbau}{sec:KonzepteundAufbau-ende};

\ref{sec:tech-Versionsverwaltung}  \vpagerefrange{sec:tech-Versionsverwaltung}{sec:tech-Versionsverwaltung-ende};

\ref{sec:impl-Versionsverwaltung}  \vpagerefrange{sec:impl-Versionsverwaltung}{sec:impl-Versionsverwaltung-ende}

\section{Motivation} \label{sec:Motivation}
Im Rahmen des Studiums an der Fachhochschule Harz in Wernigerode muss jeder Student des Studiengangs \emph{Kommunikationsinformatik} eine Projektarbeit abgeben. Dies bedeutet, da� der Student eine Aufgabe (meist Programmieraufgabe) alleine oder in einem kleinen Team bew�ltigen muss. Die Professoren der Hochschule bieten dabei viele interessante Projektarbeiten an, sind jedoch auf offen f�r eigene Vorschl�ge der Studenten.\\
Da in den Teamprojekten\footnote{Auch das Teamprojekt ist Bestandteil des Studiums. Beim Teamprojekt m��en mehrere Studenten (7-15) gemeinsam eine Programmieraufgabe umsetzen.} \emph{Labmin}\footnote{\url{http://labmin.de.vu}} und \emph{German Team Sony Aibo}\footnote{\url{http://www.der-baer.com/projects.htm}} eine interessante Aufgabe von den Studenten gel�st wurde, sollte das dort erlernte Wissen vertieft und weiter ausgebaut werden.


\section{Ideen zur Projektarbeit} \label{sec:Ideen}
\subsection{Tippspiel}\label{sec:Tippspiel}
Die erste Idee dieser Projektarbeit war die Umsetzung eines Tippspiels in Java, passend zu den damaligen Fussball-Europameisterschaft in Portugal. Diese Idee wurde im JavaMagazin\footnote{\citep{Frotscher2004}\citep{Frotscher2004a}\citep{Frotscher2004b}} in mehreren Ausgaben aufgegriffen und verschiedene Ansatzm�glichkeiten diskutiert. Die Idee unseres Tippspiel war dabei eine Webanwendung mit Datenbankanbindung. Nutzer dieses Systems sollten sich in verschiedenen Tippgemeinschaften, mit je einem Tippgemeinschaftsverwalter, zusammen finden und gemeinsam die EM 2004 tippen. Das Tippspiel sollte jedoch nicht nur auf die EM 2004 zugeschnitten sein, sondern auch f�r andere Fu�ballereignisse tauglich sein. Zus�tzlich kam von unserer Seite die Idee, eine Webanwendung zur Verwaltung der Bundesligaergebnisse. Ein Tippspielsystem sollte dann auf diese Daten zur�ckgreifen und so ein Bundesligatippspiel darstellen k�nnen.\\
Dieser Gedanke wurde jedoch aus verschiedenen Gr�nden verworfen. Zum einen war es nicht unsere Idee, sondern die des Javamagazin's und zum anderen wussten wir nicht sofort was bei diesem System alles zu realisieren war. Die grobe Funktionsweise war allen klar, jedoch fehlte bei diesem System das gewisse etwas. 
%===========================================
\subsection{Videosoftware} \label{sec:Videosoftware}
Da jeder von uns schon einmal ein Video in einer Videothek ausgeliehen, kam uns der Gedanke einer Onlinevideothek. Solche Videotheken gibt es mittlerweile wie bspw. Amango\footnote{\url{http://www.amango.de}}, Netleih\footnote{\url{http://www.netleih.de}},  Invdeo\footnote{\url{http://www.invdeo.de/}} und Verleihshop \footnote{\url{http://www.verleihshop.de}}. Bei genauer Betrachtung dieser Onlinevideotheken, fragten wir uns wie solch eine Videothek technisch funktioniert. Da wir gerade auf der Suche nach einem idealen Projekt waren, hatten wir damit eins gefunden.\\
Es solle versucht werden eine Online-Videothek mit entsprechenden Modulen zu realisieren.



\section{Zielsetzung} \label{sec:Zielsetzung}
Zielsetzung dieses Projektes ist dabei Erfahrung mit verschiedenen neuen Technologien zu sammeln und selbst�ndig an einem Projekt zu arbeiten. Sowohl die eigene Gedanken, Ideen, Planung und auch Realisierung dieses Projektes sollten uns auf eine sp�tere Eigenverantwortung im Berufsleben vorbereiten. Das Projekt sollte dabei keine vollst�ndige und fehlerfreie Implementierung darstellen. Uns war bewu�t, dass wir nur einen einfachen Prototypen einzelner Module realisieren k�nnten.\\
%===========================================



\label{sec:Einfuehrung-ende}