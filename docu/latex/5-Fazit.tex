\chapter{Fazit} \label{sec:Fazit}
Mit Hilfe dieses Projektes wurde bei allen Projektmitglieder Erfahrungen im Bereich Softwareplanung und Softwareentwicklung gesammelt. Bei diesem Projekt wurde eine eigenst�ndige Idee in die Realit�t umgesetzt, wobei alle Projektmitglieder dabei intensiv und mit viel Interesse an einer m�glichst guten Realisierung mitarbeiteten.\\

Technologisch gesehen haben sich die Projektmitglieder in Themengebiete gewagt, die ihnen bis dahin zum Teil relativ fremd waren. So wurden im Bereich vom objektrelationalen Mapping �berhaupt die ersten Erfahrungen gesammelt. Dazu wurde die Erkenntnis gewonnen, dass Hibernate ein sehr vorteilhaftes und umfassendes Werkzeug ist, um die Persistenzschicht von der Businessschicht unabh�ngig zu gestalten.\\
So wurde erreicht, dass diese Persistenzschicht vollst�ndig implementiert werden konnte und f�r den Zugriff aus dem Kundenmodul und dem Verwaltungsmodul zur Verf�gung steht.\\

Weiterhin ist festzustellen, dass die Implementierung des Projekts noch nicht abgeschlossen ist, so dass z.B. noch grundlegende Funktionalit�ten des Kundenmoduls umgesetzt werden m�ssen. Trotzdem konnte und kann hier die Verwendung des Webframeworks Apache Struts dazu beitragen, dass mit relativ wenig Aufwand und Entwicklungszeit die Entwicklung einer 3-schichtigen Webanwendung erm�glicht wird. Dies hat den Vorteil, dass Design, Businesslogik und Daten von einander getrennt dargestellt und umgesetzt werden konnten.\\

Durch die Verwendung von Subversion als Versionsverwaltung wurde der Umgang mit einer neuen Technologie schnell zur Gewohnheit. Dadurch ist es f�r die Projektmitglieder zum Alltag geworden, �nderungen sofort f�r alle zu speichern und mit aussagekr�ftigen Kommentaren zu versehen. Durch die netzwerksparende �bertragung von Daten, war auch der Einsatz von Modems m�glich. Durch den  automatischen Versand von Emails wurden andere Projektmitglieder schnell und kompakt �ber �nderungen informiert. Die installation auf einem Windowstestrechner mit einer gesicherter HTTPS-Verbindung gestaltete sich als unkompliziert, wie die Installation einer gesicherten ssh+svn Verbindung auf einem Fedore Core 3 Linux Rechner, dank gut beschriebener Anleitungen.\\

Aufgrund der SWT-Bibliothek konnte eine einfach zu bedienende Benutzeroberfl�che erstellt werden, mit der alle notwendigen Objekte f�r eine Online-Videothek einfach editiert und erstellt werden k�nnen. Aufgrund der verwendeten Reiter f�r jedes Untermen� kann schnell zwischen den einzelnen Men�s umgeschaltet und Ver�nderungen vorgenommen werden. Die verschiedenen Dialogfenster erleichtern die Eingabe f�r den Benutzer und zugleich die Bearbeitung im Quellcode.\\

Dieses Projekt k�nnte auch als Grundlage f�r andere Studentenprojekte dienen. Dabei m��en die Studenten sich nicht in dieses spezielle Projekt einarbeiten, sondern k�nnen ein selbst�ndiges Modul entwickeln. So k�nnten Studenten bspw. ein Suchmodul mit Apache Lucence entwickeln oder ein Barcodemodul zur Verarbeitung von Barcodes entwickeln. Weiterhin k�nnte ein Modul zur Erstellung von Rechnungen oder anderen Reports mit Eclipse Birt erstellt werden, ein Modul zur Verwaltung der Standorte/Lagerorte der DVD's oder ein Modul zur Berechnung der k�rzesten Wegstrecke bei der Zusammenstellung einer bzw. mehrerer Bestellungen. Jedes dieser Module k�nnte dann mit geringen Aufwand in die Online-Videothek integriert werden.




\label{sec:Fazit-ende}

