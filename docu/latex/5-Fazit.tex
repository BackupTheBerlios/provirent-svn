\chapter{Fazit} \label{sec:Fazit}
Mit Hilfe dieses Projektes wurde bei allen Projektmitglieder Erfahrungen im Bereich Softwareentwicklung und Softwareplanung gesammelt. Bei diesem Projekt wurde eine eigenst�ndige Idee in die Realit�t umgesetzt, wobei alle Projektmitglieder Ihren Teil dazu beitrugen.

Technologisch gesehen haben sich die Projektmitglieder in Themengebiete gewagt, die ihnen bis dahin zum Teil relativ fremd waren. So wurden im Bereich vom objektrelationalen Mapping �berhaupt die ersten Erfahrungen gesammelt. Dazu wurde die Erkenntnis gewonnen, dass Hibernate ein sehr vorteilhaftes und umfassendes Werkzeug ist, um die Persistenzschicht von der Businessschicht unabh�ngig zu gestalten.\\
So wurde erreicht, dass diese Persistenzschicht vollst�ndig implementiert werden konnte und f�r den Zugriff aus dem Kundenmodul und dem Verwaltungsmodul zur Verf�gung steht.

Weiterhin ist festzustellen, dass die Implementierung des Projekts noch nicht abgeschlossen ist, so dass z.B. noch grundlegende Funktionalit�ten des Kundenmoduls umgesetzt werden m�ssen. Trotzdem konnte und kann hier die Verwendung des Webframeworks Apache Struts dazu beitragen, dass mit relativ wenig Aufwand und Entwicklungszeit die Entwicklung einer 3-schichtigen Webanwendung erm�glicht wird. Dies hat den Vorteil, dass Design, Businesslogik und Daten von einander getrennt dargestellt und umgesetzt werden konnten. 




Dieses Projekt k�nnte auch als Grundlage f�r andere Studentenprojekte dienen. Dabei m�ssen die Studenten sich nicht in dieses spezielle Projekt einarbeiten, sondern k�nnen ein selbst�ndiges Modul entwickeln. So k�nnten Studenten bspw. ein Suchmodul mit Apache Lucence oder ein Barcodemodul zur Verarbeitung von Barcodes entwickeln. Weiterhin k�nnte ein Modul zur Erstellung von Rechnungen oder anderen Reports mit Eclipse Birt erstellt werden, ein Modul zur Verwaltung der Standorte/Lagerorte der DVD's oder ein Modul zur Berechnung der k�rzesten Wegstrecke bei der Zusammenstellung einer bzw. mehrerer Bestellungen.

