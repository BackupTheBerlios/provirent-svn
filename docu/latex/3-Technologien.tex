\chapter{Technologien} \label{sec:Technologien}
In diesem Abschnitt werden kurz die zu verwendeten Technologien verwendet.

	
\section{Datenbank}
Zum Einsatz soll eine OpenSource Datenbank kommen. Gedanken an eine kommerzielle Datenbank kam aus Gr�nden der Lizenzkosten nicht auf. \\
Zu Auswahl standen mehrere OpenSource Datenbanken. MySql \footnote{\url{http://dev.mysql.com/downloads/mysql/4.0.html}}, SAP DB \footnote{\url{http://dev.mysql.com/downloads/maxdb/7.5.00.html}} , HSQL DB \footnote{\url{http://hsqldb.sourceforge.net}} und Firebird \footnote{\url{http://firebird.sourceforge.net}}.
		\begin{itemize}
			\item \textbf{MySql}
				\begin{itemize}
					\renewcommand{\labelitemii}{+}
					\item sehr verbreitet
					\item einige Erfahrung
					\item gut Dokumentiert \& gro�e Community
					\item 
				\end{itemize}
				
				\begin{itemize}
					\item schlechtes Lizenzmodell
					\item zu bekannt
					\item keine Trigger
					\item meist nur im privat bzw. klein Unternehmer Einsatz
				\end{itemize}
				
			\item \textbf{SAP DB}
				\begin{itemize}
					\renewcommand{\labelitemii}{+}				
					\item 
					\item Datenbank seit mehreren Jahren bei SAP im Einsatz
				\end{itemize}
				
				\begin{itemize}
					\item schlechte Skalierbarkeit, da der Datenbank Speicherbereich im Vorfeld festgelegt werden muss
					\item schlechte Erfahrung
				\end{itemize}
				
			\item \textbf{HSQLDB}
				\begin{itemize}
					\renewcommand{\labelitemii}{+}				
					\item reine JavaDatenbank
					\item sehr klein
					\item kann als reine Speicher Datenbank verwendet werden (Daten nur im Arbeitsspeicher)
					\item kann als Applikations Datenbank verwendet werden (nur eine Applikation benutzt die Datenbank)
				\end{itemize}
				
				\begin{itemize}
					\item nicht f�r gro�e Applikationen geeignet
					\item 
				\end{itemize}			

			\item \textbf{Firebird}
				\begin{itemize}
					\renewcommand{\labelitemii}{+}				
					\item geringe Erfahrung durch Studium
					\item sehr klein
					\item gute grafische Tools
					\item Original Sourcen kommen von Borland
					\item Interbase Datenbank seit mehreren Jahren im Professionelle einsatz
				\end{itemize}
				
				\begin{itemize}
					\item schlechtes Lizenzmodell
					\item 
				\end{itemize}
		
		\end{itemize}
		
Wir haben uns f�r die Firebird Datenbank entschieden, da es keine wirkliche Konkurrenz im Open Source Bereich gibt.\\
HSQL scheidet schon aus, weil es nicht f�r grosse Datenmengen geeignet ist. Bei der SAP DB muss der ben�tigte Speicherplatz der Datenbank vorher bekannt sein, was bei unserem Projekt nicht der Fall ist. MYSQL unterst�tzt keine Triggers und ist zu bekannt, d.h. MySql kann und sollte jeder Informatiker kennen und benutzt haben. \\
Firebird ist f�r uns relativ neu und die Erfahrungen die wir in der Vorlesung "`Datenmanagment 2"' bekommen haben, war sehr positiv. Da diese Datenbank urspr�nglich von Borland kommt, ist diese Datenbank auch nicht so neu, wie viele Denken.\\
Es soll aber schon am Anfang des Projektes bedacht werden, dass die Datenbank zu einem sp�teren Zeitpunkt eventuell mit einer professionelle Datenbank\footnote{z.B. DB2 von IBM} ausgetauscht werden k�nnte. Deswegen muss schon am Anfang eine hohe Abstraktionsebene vorhanden sein, so dass eventuelle Datenbankspezifische Elemente (Klassen) sehr einfach ausgetauscht werden k�nnen.
		
		
				
\section{Versionsverwaltung}
		\subsection{Concurrent Versions System - CVS}
		\subsection{Subversion}
		
\section{Entwicklungsumgebung}
		\subsection{JBuilder}
		\subsection{Netbeans}
		\subsection{Eclipse}

\section{grafischen Benutzerschnittstellen in Java}
		\subsection{Abstract Window Toolkit - AWT}
		\subsection{Swing}
		\subsection{Standard Widget Toolkit - SWT}
		
\section{Java-Web-Anwendungen}
		\subsection{Java Server Faces - JSF}
		\subsection{Struts}

\section{Persistenzschichten in Java}
		\subsection{Java Data Objects}
		\subsection{Hibernate}
		
		
		
		

		
\section*{Datenbank - erste Ideen}	\label{asas}




	\begin{itemize}
			
  		\item Kunden
  		\begin{itemize}
  			\setlength{\itemsep}{-1ex plus0.5ex minus0.3ex}
  			\item kundenid \emph{Integer Autoincrement Primary Key}
  			\item name \emph{VARCHAR(200)}
  			\item vorname \emph{VARCHAR(200)}
  			\item strasse \emph{VARCHAR(200)}
  		\end{itemize}
  	 \item benutzer
  		\begin{itemize}
  			\setlength{\itemsep}{-1ex plus0.5ex minus0.3ex}
  			\item benutzerid \emph{Integer Autoincrement Primary Key}
  		\end{itemize}

  	 \item dvds
  		\begin{itemize}
  			\setlength{\itemsep}{-1ex plus0.5ex minus0.3ex}
  			\item dvdid \emph{Integer Autoincrement Primary Key}
  		\end{itemize}

  	 \item genre
  		\begin{itemize}
  			\setlength{\itemsep}{-1ex plus0.5ex minus0.3ex}
  			\item genreid \emph{Integer Autoincrement Primary Key}
  		\end{itemize}

  	 \item artikel
  		\begin{itemize}
  			\setlength{\itemsep}{-1ex plus0.5ex minus0.3ex}
  			\item artikelid \emph{Integer Autoincrement Primary Key}
  		\end{itemize}

  	 \item verleih
  		\begin{itemize}
  			\setlength{\itemsep}{-1ex plus0.5ex minus0.3ex}
  			\item verleihid \emph{Integer Autoincrement Primary Key}
  		\end{itemize}

  	 \item preis
  		\begin{itemize}
  			\setlength{\itemsep}{-1ex plus0.5ex minus0.3ex}
  			\item preisid \emph{Integer Autoincrement Primary Key}
  		\end{itemize}
  	 

   \end{itemize}			

	\section*{Prototyp}
	Bevor versuchen ein fertiges Produkt zu realisieren und daran vermutlich scheitern werden, haben wir beschlossen einen einfachen Prototyp zu programmieren. Dieser soll die wichtigsten Merkmale besitzen und zu Demostrationszwecken dienen.
	Jedoch soll es auch m�glich sein, diesen Prototypen zu einem fertigen Produkt fertig zu entwickeln. Der Prototyp soll also nicht quick \& dirty programmiert werden. \\
	Der Prototyp soll haupts�chlich die Kundenseite implementieren. D.h. er soll eine Webanwendung bereitstellen, bei der der Kunde bzw. Interessent sich regiestrieren kann und die Videothek benutzen kann. Dies bedeutet er kann sich die vorhanden Videos anschauen (Informationen zu diesen), kann sich die Verf�gbarkeit anschauen, Video ausleihen, Rechnungen ansehen bzw. ausdrucken und eine Liste mit all seinen bisherigen Bestellungen anschauen. \\
	Das Modul f�r die Verwaltung der DVD's ist in diesem Prototypen noch nicht vorgesehen. \\
	Das Modul f�r das Versenden und Empfangen ist nur in einfacher Variante vorgesehen. Der Mitarbeiter der Onlinevideothek bekommt eine kleine Anwendung auf Konsolenbasis ohne grafische Oberfl�che.
		
		