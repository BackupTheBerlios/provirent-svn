\section{Versionsverwaltung mit Subversion} \label{sec:impl-Versionsverwaltung}
Als Versionsverwaltung wurde Subversion verwendet. Dies hat mehrere Gr�nde: Zum einen war Subversion zu diesem Zeitpunkt eine neue und z.T. auch unbekannte Technologie, die somit ihren Reiz hatte. Zum anderen wurde Subversion als Nachfolger von CVS angepriesen und sollte viele Nachteile eliminieren. Eins der wichtigen Vorteile von Subversion ist die geringe Netzwerklast. Bei einem "`Commit"' werden nur die Unterschiede zur Vorg�ngerversion �bertragen und nicht wie bei CVS jede Datei komplett. Um Subversion verwenden zu k�nnen, wurde ein Rechner mit einem installieren Subversion Server ben�tigt, der m�glichst 24h im Internet verf�gbar ist. Um mit einem Client auf einen Subversion Server zuzugreifen, gibt es verschiedene �bertragungsprotokolle: Ein eigenes Subversion protokoll (svn), eine Kombination aus Secure Shell und dem eigenen Protokoll (ssh+svn), �ber http oder �ber https. Die sicherste Methode �ber ssh+svn, jedoch erfordert diese die Installation eines SSH Server, was unter Linux, dank verschiedener Anleitungen, einfach geht, jedoch unter Windows nicht so einfach zu realisieren ist. Unter Windows wurde zu Testzwecken ein Apache2 Webserver mit einem Zertifikat zur �bertragung von Daten per https installiert. Dieser Webserver wurde mit dem Subversion Server kombiniert und jeglicher Datentransfer f�r Subversion erfolgte �ber https und dessen eingestellen Port. Der Nachteil war jedoch, dass keiner die M�glichkeit hatte, einen PC 24h online zur Verf�gung zu stellen. So wurde nach einiger Suche im Internet der Open Source Anbieter berlios.de entdeckt. Dort konnten Open-Source Projekte ihre Quellcodes in einer Subversion Versionsverwaltung mittels ssh+svn kostenlos hosten.\\





\texttt{\textbf{maximal zwei Seiten}}


Befehl zum Erzeugen eines neues Repository.
Zugangsbeschr�nkungen
Clientauswahl und Verwendung mit Screenshoots.
Typischer Aufbau eines Root-Verzeichnisses (trink,branches,tags)
Tags erstellen
mit TortoiseSVn nur ein bestimmtes Verzeichniss auschecken, sichbar machen.
automatischer Versand von Emails, nach dem Commit.







%Hier danach nicht mehr schreiben
\label{sec:impl-Versionsverwaltung-ende}