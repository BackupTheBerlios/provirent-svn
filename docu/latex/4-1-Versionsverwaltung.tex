\section{Versionsverwaltung mit Subversion} \label{sec:impl-Versionsverwaltung}

Es wurde Subversion verwendet, da es eine neue Technologie ist, zu diesem zeitpunkt relativ unbekannt war. Weiterhin wurde es als Nachfolger von CVS "`verkauft"' und bot im Vergleich zu CVS einige Vorteile in Bezug auf Geschwindigkeit und Netztwerklast.
Nun war die Frage wo und wie einen Subversion Server installieren. Linux war nicht m�glich, da keiner ausreichend Linux-kenntinisse besitzt, also einen Windows Server.
Mit Subversion gibt es verschiedene M�glichkeiten um mit dem Server zu kommunizieren. den eigene Subversion Server und das eigene svn protokoll. Nachteil ist, dass dies nicht Verschl�sselt ist. Die sicherste methode ist �ber ssh+svn, eine Kombination aus der Secure Shell (bekannt aus Linux) und dem Subversion Server. Der Nachteil hier ist wieder Windows. F�r linux ist solch ein setup nicht kompliziert, da es einfache Schritt-f�r-Schritt Anleitungen gibt. Unter Windows gibt es zwar Secure Shell Server, jedoch keine Anleitungen bzw. Hinweise, Tipps wie man diese mit dem Subversion Server integriert.



maximal zwei Seiten
		







%Hier danach nicht mehr schreiben
\label{sec:impl-Versionsverwaltung-ende}