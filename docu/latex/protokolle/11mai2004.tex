\chapter{Protokoll vom 11. Mai 2004}

Drei Frameworks stehen zur Auswahl \\
\begin{itemize}
	\item Apache Cocoon
	\item Apache Struts
	\item Apache Tapestry
\end{itemize}

Jeder erstellt eine einfache (Web)Anwendung mit Hilfe eines dieser Frameworks 
folgende Komponenten sollen/muessen enthalten sein: \\
\begin{itemize}
	\item einfache LoginSeite (�ber Datenbank)
	\item Liste aller Videos in Datenbank anzeigen
	\item EingabeMaske f�r neues Labor
	\item Validierung der Eingabedaten
	\item dynamische Navigation
	\item eventuell ein Bild f�r den Status der einzelnen Bilder (dynamisches Bild??)
\end{itemize}

F�r diese BeispielAnwendung sollen m�glichst viele Elemente des jeweiligen Framework verwendet werden.
Wichtig ist dabei der Umgang und die Bedienbarkeit des Systems. \\
Wiederverwendbarkeit einzelner Module. \\
Design und Logik Trennung vorhanden? Kann das Design einfach/schnell ausgetauscht werden.\\
\\
Es geht dabei nicht um ein 100% fehlerfreies System.\\
Design spielt keine wichtige Rolle, es sollte jedoch beachtet werden, dass dieses sp�ter vom Kunden ausgetauscht werden m�chte.\\
\\
\\
\begin{itemize}
	\item Struts: Stefan
	\item Cocoon: Philipp
	\item Tapestry: Remo
\\
	\item Namen f�r das Projekt finden
	\item Link mit Beispiel Webseiten rumschicken
\end{itemize}